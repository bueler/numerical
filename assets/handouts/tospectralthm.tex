\documentclass[12pt]{article}
\usepackage[top=1.2in, bottom=0.8in, left=1.0in, right=1.2in]{geometry}

\usepackage{graphicx,color,enumitem,verbatim,fancyvrb}
\usepackage{amsmath,amsthm,amsbsy,amssymb,bm}
\usepackage{palatino}
\usepackage{mdframed}

\usepackage{tikz}
\usepackage[colorlinks=true]{hyperref}

\makeatletter

%% The following commands put defined left and right headers on the top, and a page number
%% on the bottom of all pages beyond page 1
\usepackage{fancyhdr}
\pagestyle{fancy}
\fancyfoot[C]{\ifnum \value{page} > 1\relax\thepage\fi}
\fancyhead[L]{\ifx\@doclabel\@empty\else\@doclabel\fi}
\fancyhead[R]{\ifx\@docdate\@empty\else\@docdate\fi}
\headheight 15pt
\def\doclabel#1{\gdef\@doclabel{#1}}
\def\docdate#1{\gdef\@docdate{#1}}
\makeatother

%% General formatting parameters
\parindent 0pt
\parskip 6pt plus 1pt

\newcommand{\bA}{\mathbf{A}}
\newcommand{\bC}{\mathbf{C}}
\newcommand{\bI}{\mathbf{I}}
\newcommand{\bX}{\mathbf{X}}

\newcommand{\bzero}{\bm{0}}

\newcommand{\cC}{\mathcal{C}}
\newcommand{\cD}{\mathcal{D}}
\newcommand{\cH}{\mathcal{H}}
\newcommand{\cL}{\mathcal{L}}
\newcommand{\cM}{\mathcal{M}}
\newcommand{\cS}{\mathcal{S}}
\newcommand{\cV}{\mathcal{V}}
\newcommand{\cW}{\mathcal{W}}

\newcommand{\CC}{\mathbb{C}}
\newcommand{\QQ}{\mathbb{Q}}
\newcommand{\RR}{\mathbb{R}}
\renewcommand{\SS}{\mathbb{S}}

\newcommand{\ip}[2]{\left<#1,#2\right>}

\newcommand{\range}{\operatorname{range}}
\newcommand{\Span}{\operatorname{span}}

\newcommand{\eps}{\epsilon}
\newcommand{\ds}{\displaystyle}

\newcommand{\sect}[1]{\subsection*{#1.}}

\newcommand{\defin}{\emph{Definition.}\,\,}
\newcommand{\lem}{\emph{Lemma.}\,\,}
\newcommand{\thm}{\emph{Theorem.}\,\,}

\newlist{enumex}{enumerate}{3}
\setlist[enumex]{label={Example \Alph*.},leftmargin=27mm,before=\raggedright}



\doclabel{Math 617 Functional Analysis}
\docdate{March 2024 (Bueler)}

\begin{document}
\strut
\centerline{{\Large \textbf{Definitions and facts}}}

\large \smallskip
\centerline{{\textbf{(leading to the spectral theorem)}}}
\bigskip

\normalsize
Page numbers are given, from Borthwick, \emph{Spectral Theory}.  Notation: $\forall$=``for all'', $\exists$=``there exists'', $\cH$ is a separable Hilbert space, $T$ is an (unbounded) operator on $\cH$, $U \in \cL(\cH)$ is a unitary operator, and $A$ is an (unbounded) self-adjoint operator on $\cH$.
%\sect{Metric spaces} \label{topic:metric}


\newcommand{\itwo}[2]{{\small \textbf{#1}} {\footnotesize p #2} \,\,}
\newcommand{\df}[1]{\,\itwo{def}{#1}}
\newcommand{\ft}[1]{\itwo{\underline{fact}}{#1}}

\begin{itemize}[leftmargin=10mm,itemsep=0mm]
\item[\df{36}] an \emph{operator} $T$ is a linear map on $\cH$ with a dense domain $\cD(T)$
\item[\df{38}] the \emph{adjoint} of $T$ is an operator $T^*$, with domain
	$$\cD(T^*) = \left\{v\in\cH\,:\,\ell(u)=\ip{v}{Tu} \in \cL(\cH,\CC)\right\},$$
so that $\ip{T^* v}{u} = \ip{v}{Tu}$ \, for all $v\in\cD(T^*)$, $u\in\cD(T)$
\item[\df{41}] an operator is \emph{closed} if its graph is a closed subset of $\cH\times \cH$
\item[\ft{43}] the adjoint $T^*$ is always closed
\item[\ft{44}] $T=T^{**}$ if $T$ is closed
\item[\ft{44}] $T$ closable $\iff$ $\cD(T^*)$ dense
\item[\ft{44}] \emph{closed graph thm.} when $\cD(T)=\cH$: \, $T$ closed $\iff$ $T\in\cL(\cH)$
\item[\df{46}] $T$ has \emph{bounded inverse}: \, $\exists$ $T^{-1}\in\cL(\cH)$ s.t.~$TT^{-1}=I$ on $\cH$ and $T^{-1}T=I$ on $\cD(T)$
\item[\ft{46}] $T^{-1}\in\cL(\cH)$ $\iff$ $T$ is closed, $T$ is bounded away from zero, and $\range(T)$ dense
\item[\df{47}] $A$ is \emph{self-adjoint} if $A^*=A$ \hfill $\leftarrow$ \textbf{requires} $\cD(A^*)=\cD(A)$
\item[\df{67}] \emph{eigenvalue} and \emph{eigenvector}: \, $T\phi=\lambda\phi$ \,for $\phi\in\cD(T)\setminus\{0\}$ and $\lambda\in\CC$
\item[\df{68}] \emph{spectrum}: the set $\sigma(T)=\{\lambda \in\CC\,:\,T - \lambda \text{ does not have a bounded inverse}\}$
\item[\df{68}] \emph{resolvent set}: $\rho(T) = \CC\setminus \sigma(T)$
\item[\df{68}] if $\lambda\in\rho(T)$ then $R_\lambda = (T-\lambda)^{-1}$ is the \emph{resolvent} operator
\item[\ft{68}] $\sigma(T)=\CC$ if $T$ is not closed
\item[\ft{69}] $\sigma(T) \subset B_{\|T\|}(0)$ if $T$ is bounded
\item[\ft{69}] $\sigma(T^*) = \sigma(T)^*$, $\rho(T^*)=\rho(T)^*$, and $\left[(T-z)^{-1}\right]^* = \left(T-\overline{z}\right)^{-1}$
\item[\ft{71}] for  $f:X\to\CC$ measurable and $M_f$ a multiplication operator on $L^2(X,d\mu)$:

$\lambda\in\CC$ is an eigenvalue of $M_f$ $\iff$ $\mu(f^{-1}(\lambda))>0$
\item[\df{71}] $\operatorname{ess-range} f = \left\{z\in\CC\,:\,\mu(f^{-1}(B_\eps(z))) > 0 \, \forall \eps>0\right\}$
\item[\ft{71}] $\sigma(M_f) = \operatorname{ess-range} f$
\item[\ft{71}] $\|(M_f - z)^{-1}\| = \Big(\operatorname{dist}\big(z,\sigma(M_f)\big)\Big)^{-1}$
\item[\ft{83}] if $T$ closed then $\rho(T)$ is open and $R_z = (T-z)^{-1}$ is analytic in $z$ on $\rho(T)$
\item[\df{85}] \emph{spectral radius}: $\ds r(T) = \sup_{z \in \sigma(T)} |z|$
\item[\ft{85}] if $T$ bounded then $r(T) \le \|T\|$
\item[\ft{86}] $\sigma(A)\subset \RR$
\item[\ft{87}] $z\in\sigma(A)$ $\iff$ $\exists \{u_n\}\subset \cD(A) \text{ s.t.~} \|u_n\|=1 \text{ and } \|(A-z)u_n\| \to 0$
\item[\df{17}] $U$ is \emph{unitary} if it is bijective and an isometry (i.e.~$\|Ux\|=\|x\|\,\forall x\in\cH$)
\item[\ft{17}] $U$ unitary $\iff$ $U$ bijective \& $\ip{Ux}{Uy}=\ip{x}{y}\,\forall x,y\in\cH$
\item[\ft{102}] $U$ unitary $\iff$ $U\in\cL(\cH)$ and $UU^* = U^*U=I$
\item[\df{102}] \emph{functional calculus} means we can apply $f:\CC\to\CC$ to create operator $f(T)$
\item[\df{102}] $\SS = \{z\in\CC\,:\,|z|=1\}$
\item[\df{103}] \emph{positive} FIXME
\item[\ft{103}] \textbf{Continuous functional calculus for unitaries.} given $U$ unitary there is a map $C(\SS) \to \cL(\cH)$, $f\mapsto f(U)$ so that
\renewcommand{\labelenumi}{(\alph{enumi})}
\begin{enumerate}
\item[(0)] $f(U) = I$ if $f(z) = 1$
\setcounter{enumi}{0}
\item $f(U)^*=\overline{f}(U)$
\item $f(U)g(U) = (fg)(U)$
\item if $f\ge 0$ then $f(U)\ge 0$ FIXME
\end{enumerate}
\item[\df{}] 
\item[\ft{}] 
\end{itemize}
\end{document}
