\documentclass[12pt]{article}
\usepackage[top=1.2in, bottom=0.8in, left=1.0in, right=1.2in]{geometry}

\usepackage{graphicx,color,enumitem,fancyvrb}
\usepackage{amsmath,amsthm,amsbsy,amssymb}
\usepackage{palatino}
\usepackage{mdframed}

\usepackage{tikz}
\usepackage[colorlinks=true]{hyperref}

\makeatletter

%% The following commands put defined left and right headers on the top, and a page number
%% on the bottom of all pages beyond page 1
\usepackage{fancyhdr}
\pagestyle{fancy}
\fancyfoot[C]{\ifnum \value{page} > 1\relax\thepage\fi}
\fancyhead[L]{\ifx\@doclabel\@empty\else\@doclabel\fi}
\fancyhead[R]{\ifx\@docdate\@empty\else\@docdate\fi}
\headheight 15pt
\def\doclabel#1{\gdef\@doclabel{#1}}
\def\docdate#1{\gdef\@docdate{#1}}
\makeatother

%% General formatting parameters
\parindent 0pt
\parskip 6pt plus 1pt

\newcommand{\bA}{\mathbf{A}}
\newcommand{\bC}{\mathbf{C}}
\newcommand{\bI}{\mathbf{I}}
\newcommand{\bX}{\mathbf{X}}

\newcommand{\CC}{\mathbb{C}}
\newcommand{\RR}{\mathbb{R}}

\newcommand{\exer}[1]{\noindent \textbf{Problem #1.} \,}
%\newcommand{\epart}[1]{\noindent \textbf{(#1)} \,}
\newcommand{\eps}{\epsilon}
\newcommand{\ds}{\displaystyle}

%\newcommand{\sect}[1]{\medskip\noindent \textbf{#1.}}
\newcommand{\sect}[1]{\subsection*{#1.}}

\doclabel{Math 617}
\docdate{January 2024; Bueler}


\begin{document}
\strut
\centerline{{\Large \textbf{Handout: Definitions and facts}}}

\centerline{{\Large\strut \textbf{which you will need to get started}}}
\bigskip

Functional analysis is the study of vector spaces which have a topology.  Therefore studying functional analysis requires you to have some sense of what a ``topology'' is.  Our textbook (D.~Borthwick, \emph{Spectral Theory: Basic Concepts and Applications}, GTM 284, Springer, 2020) assumes that you know about basic topology.

\sect{Metric spaces}

To talk about topology we start with a ``metric'' and then to define open and closed sets from that.

\newcommand{\defin}{\noindent \emph{Definition.}\,}

\defin Suppose $X$ is any set of points.  A function $d:X\times X\to \RR$ is a \textbf{metric} if, for all $x,y,z\in X$, these conditions hold:
\begin{enumerate}
\item $d(x,x)=0$, and $x\ne y$ implies $d(x,y)>0$
\item $d(x,y)=d(y,x)$ \hfill (symmetry)
\item $d(x,z) \le d(x,y) + d(y,z)$ \hfill (triangle inequality)
\end{enumerate}

Note that by condition 1 a metric always produces nonnegative numbers.  Observe that one cannot substitute complex numbers for real numbers in this definition because an ordering ``$\le$'' is then not available.  The addition symbol in condition 3 acts on real numbers, and not on the elements of the general set $X$.  (We may not be able to add elements of $X$!)

If you have seen ``norms'' on vector spaces then the above definition will ring a bell; more soon!

\defin If $X$ is a set and $d$ is a metric then one calls the pair $(X,d)$ a \textbf{metric space}.

Here are two examples of metric spaces:

\newlist{enumex}{enumerate}{3}
\setlist[enumex]{label={Example \Alph*.},leftmargin=27mm,before=\raggedright}

\newcommand{\doit}[2]{\emph{\underline{Instructions:}\, #2} \vspace*{#1mm}}
\newcommand{\showit}[2]{\emph{\underline{Show it!} (#2)} \vspace*{#1mm}}

\begin{enumex}
\item Let $X=S^1$ be the unit sphere, namely the set of points in $\RR^n$ which are Euclidean distance 1 from the origin.  Let $d$ be the Euclidean distance between points of $S^1$.

\showit{20}{Argue that $d$ defines a metric.}

\clearpage
\vspace*{20mm}

\item Let $X$ be any set whatsoever.  Define $d(x,x)=0$ and $d(x,y)=1$ if $x\ne y$.

\showit{50}{Argue that $d$ defines a metric.  Note $X$ could be very general, such as the set of all words in the English language.}
\end{enumex}


\sect{Vector spaces}  Many metric spaces are actually vector spaces with norms.  I think you already know what an abstract vector space is, so here the definition is an exercise.

\doit{0}{Fill in the axioms for a vector space, to complete the definition below.  Add bullet points as needed.}

\defin A set $X$ (of vectors) with an operation $*:\CC\times X \to X$ and another operation $+:X\times X\to X$ is a \textbf{vector space} if the following hypotheses and conditions hold:
\begin{itemize}
\item \phantom{x} \vspace{5mm}

\item \phantom{x} \vspace{5mm}

\item \phantom{x} \vspace{5mm}

\item \phantom{x} \vspace{5mm}

\item \phantom{x} \vspace{5mm}

\item \phantom{x} \vspace{5mm}

\item \phantom{x} \vspace{5mm}

\item \phantom{x} \vspace{5mm}

\end{itemize}


\sect{Norms}  
\sect{Open and closed sets}

\sect{Limits and continuity}

\sect{Compact sets}

\sect{Cauchy sequences and completeness}

\sect{Extreme value theorem}

\sect{Linear map}

\sect{Linear independence}

\sect{Measure}

\sect{Integral}

\end{document}
