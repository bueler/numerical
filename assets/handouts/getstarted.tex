\documentclass[12pt]{article}
\usepackage[top=1.2in, bottom=0.8in, left=1.0in, right=1.2in]{geometry}

\usepackage{graphicx,color,enumitem,fancyvrb}
\usepackage{amsmath,amsthm,amsbsy,amssymb}
\usepackage{palatino}
\usepackage{mdframed}

\usepackage{tikz}
\usepackage[colorlinks=true]{hyperref}

\makeatletter

%% The following commands put defined left and right headers on the top, and a page number
%% on the bottom of all pages beyond page 1
\usepackage{fancyhdr}
\pagestyle{fancy}
\fancyfoot[C]{\ifnum \value{page} > 1\relax\thepage\fi}
\fancyhead[L]{\ifx\@doclabel\@empty\else\@doclabel\fi}
\fancyhead[R]{\ifx\@docdate\@empty\else\@docdate\fi}
\headheight 15pt
\def\doclabel#1{\gdef\@doclabel{#1}}
\def\docdate#1{\gdef\@docdate{#1}}
\makeatother

%% General formatting parameters
\parindent 0pt
\parskip 6pt plus 1pt

\newcommand{\bA}{\mathbf{A}}
\newcommand{\bC}{\mathbf{C}}
\newcommand{\bI}{\mathbf{I}}
\newcommand{\bX}{\mathbf{X}}

\newcommand{\cV}{\mathcal{V}}

\newcommand{\CC}{\mathbb{C}}
\newcommand{\RR}{\mathbb{R}}

%\newcommand{\exer}[1]{\noindent \textbf{Problem #1.} \,}
%\newcommand{\epart}[1]{\noindent \textbf{(#1)} \,}
\newcommand{\eps}{\epsilon}
\newcommand{\ds}{\displaystyle}

%\newcommand{\sect}[1]{\medskip\noindent \textbf{#1.}}
\newcommand{\sect}[1]{\subsection*{#1.}}

\newcommand{\defin}{\emph{Definition.}\,\,}
\newcommand{\thm}{\emph{Theorem.}\,\,}

\newlist{enumex}{enumerate}{3}
\setlist[enumex]{label={Example \Alph*.},leftmargin=27mm,before=\raggedright}

\newcommand{\exer}[2]{\emph{\underline{Exercise.}\, #2} \vspace*{#1mm}}
\newcommand{\instruct}[2]{\emph{\underline{Instructions:}\, #2} \vspace*{#1mm}}
\newcommand{\showit}[2]{\emph{\underline{Show it!} (#2)} \vspace*{#1mm}}



\doclabel{Math 617 Functional Analysis}
\docdate{January 2024 (Bueler)}

\begin{document}
\strut
\centerline{{\Large \textbf{Handout: Definitions and facts}}}

\centerline{{\Large\strut \textbf{(which you will need to get started)}}}
\bigskip

Functional analysis is the study of vector spaces which have a topology.  Therefore you need to have some sense of what a ``topology'' is.  In fact, our textbook (D.~Borthwick, \emph{Spectral Theory: Basic Concepts and Applications}, GTM 284, Springer, 2020) assumes that you know the basics of topology, vector spaces, measures, and integrals.  This handout is an attempt to get you up to speed on these topics, so that you can read the book with real understanding.  Note that good mathematicians regularly stop reading and ask themselves ``do I understand the definition of this term?''


\sect{Metric spaces} \label{topic:metric}

We start with a ``metric'', and then define open and closed sets from that, which is a topology.  A metric is a generalized distance function.

\defin Suppose $X$ is any set.  (Elements of $X$ will be called ``points''.)   A function $d:X\times X\to \RR$ is a \textbf{metric} if, for all $x,y,z\in X$, these conditions hold:
\begin{enumerate}
\item $d(x,y)\ge 0$, and $d(x,y)=0$ if and only if $x=y$
\item $d(x,y)=d(y,x)$ \hfill (symmetry)
\item $d(x,z) \le d(x,y) + d(y,z)$ \hfill (triangle inequality)
\end{enumerate}

The addition symbol in condition 3 acts on real numbers, and not on the elements of the general set $X$.  (We may not be able to add elements of $X$!)  Also observe that one cannot substitute $\CC$ for $\RR$ in this definition because a reasonable ordering ``$\le$'' is not available for $\CC$.  If you have seen ``norms'' on vector spaces then the above definition will seem familiar; in which case see page \pageref{topic:norms}.

\defin If $X$ is a set and $d$ is a metric then one calls the pair $(X,d)$ a \textbf{metric space}.

Here are two examples of metric spaces:

\begin{enumex}
\item Let $X=S^1$ be the unit sphere, namely the set of points in $\RR^n$ which are Euclidean distance 1 from the origin.  Let $d$ be the Euclidean distance between points of $S^1$.

\showit{20}{Argue that $d$ defines a metric.}

\clearpage
\vspace*{20mm}

\item Let $X$ be any set whatsoever.  Define $d(x,x)=0$ and $d(x,y)=1$ if $x\ne y$.

\showit{50}{Argue that $d$ defines a metric.  Note $X$ could be a very general set.}
\end{enumex}


\sect{Vector spaces}  Many metric spaces are actually vector spaces with norms.  Let us start with the definition of an abstract vector space, but treat it as an exercise.  Do you remember all the conditions?

\instruct{0}{Fill in the axioms for a vector space, to complete the definition below.  Add bullet points as needed.}

\defin A set $\cV$ (of \textbf{vectors}) with a \textbf{scalar multiplication} operation $*:\CC\times \cV \to \cV$ and a \textbf{vector addition} operation $+:\cV\times \cV\to \cV$ is a \textbf{(complex) vector space} if the following hypotheses and conditions hold:
\begin{itemize}
\item \phantom{x} \vspace{5mm}

\item \phantom{x} \vspace{5mm}

\item \phantom{x} \vspace{5mm}

\item \phantom{x} \vspace{5mm}

\item \phantom{x} \vspace{5mm}

\item \phantom{x} \vspace{5mm}

\item \phantom{x} \vspace{25mm}

\end{itemize}


\sect{Norms} \label{topic:norms}

Section 2.1 of the textbook defines a ``norm''.

\instruct{0}{Fill in the definition below.  Refer to Section 2.1 as needed.}

\defin A \textbf{norm} on a complex vector space $\cV$ is a function $\|\cdot\|:\cV \to \RR$ satisfying, for all $v,w\in\cV$ and $a\in \CC$,
\begin{enumerate}
\item \phantom{foo} \vspace{5mm}

\item \phantom{foo} \vspace{5mm}

\item \phantom{foo} \vspace{7mm}

\end{enumerate}

\defin If $\cV$ is a vector space and $\|\cdot\|$ is a norm on it then we say $(\cV,\|\cdot\|)$ is a \textbf{normed vector space}.

Every normed vector space is also a metric space.  (The next Exercise asks you to show this.)  A norm is similar to a metric, but a norm requires more structure.  That is, you need to be able to add the elements of the set itself, and multiply them by scalars; the elements of the set have to be vectors and not just ``points''.

\exer{50}{Show that if $\|\cdot\|$ is a norm on $\cV$ then the distance function defined in Section 2.1, namely $\operatorname{dist}(v,w) := \|v-w\|$, is a metric, so $(\cV,\operatorname{dist})$ is a metric space.}


\sect{Open and closed sets}

Once you have a metric (or a norm) then you can define ``open'' and ``closed'' subsets, and talk about the ``boundaries'' of sets.  Our starting point is a ``ball'' around a point.

\defin Suppose $(X,d)$ is a metric space.  If $x\in X$ is a point and $\eps>0$ is a real number then the \textbf{open ball} of radius $\eps>0$ around $x$ is the set
	$$B_\eps(x) = \left\{y\in X\,:\,d(x,y) < \eps\right\}$$ 

\defin Suppose $(X,d)$ is a metric space.
\begin{itemize}
\item A subset $Y\subset X$ is \textbf{open} if for all $y\in Y$ there is $\eps>0$ so that $B_\eps(y) \subset Y$.
\item A subset $Y\subset X$ is \textbf{closed} if $X \setminus Y$ is open.
\end{itemize}

\exer{60}{Suppose $(\cV,\|\cdot\|)$ is a normed vector space.  Rewrite the above definitions using the norm.}

\exer{60}{Sketch an open set in the plane $\RR^2$, and illustrate the definition of open set.}

\clearpage\newpage
\exer{60}{Generally speaking, sets are neither open nor closed.  Sketch an example of such a set in the plane $\RR^2$.}

\defin Suppose $(X,d)$ is a metric space and $Y \subset X$ is any subset.  A point $z\in X$ is in the \textbf{boundary of } $Y$ if for all $\eps>0$, the intersection $B_\eps(z)\cap Y$ is non-empty and the intersection $B_\eps(z)\cap (X\setminus Y)$ is also non-empty.  We write $\partial Y$ for the set of points which are in the boundary of $Y$.

\exer{1}{Using a different color, sketch $\partial Y$ onto your above example.}

Finally, here is a promised definition.

\defin Suppose $X$ is a set, and if $\{Y_\alpha\}$ is a collection of subsets of $X$ which we call the ``open sets'', then we say $\{Y_\alpha\}$ is a \textbf{topology} on $X$.

In fact, the above definition is too informal.  The collection of subsets must satisfy certain requirements to qualify as a system of open sets.  The collection must include the empty set $\emptyset$ and the whole set $X$, it must be closed under arbitrary unions, and the intersection of any pair of subsets in the collection must be in the collection.  These conditions all hold if you define open sets via a metric, as we have done above.

From the definition of a ``topology'' we can make sense of a phrase like ``functional analysis is the study of vector spaces which have a topology.''  Having a system of identified open sets makes it possible to talk about limits, continuity, and so on, and ultimately to find solutions to hard problems.


\sect{Limits and continuity}

From the definition of open sets, or more directly by using open balls and/or a metric, we can define the limit of a sequence.

\defin Suppose $(X,d)$ is a metric space and $(x_n)$ is a sequence of points from $X$.  We say that \textbf{the limit of $(x_n)$ is $\hat x$}, written
	$$\lim_{n\to\infty} x_n = \hat x, \qquad \text{or} \qquad x_n \to \hat x,$$
if $\hat x \in X$ and
\begin{enumerate}
\item for each open set $Y$ which contains $\hat x$ there is $N$ so that $n\ge N$ implies $x_n \in Y$, or
\item for each $\eps>0$ there is $N$ so that $n\ge N$ implies $x_n \in B_\eps(\hat x)$.
\end{enumerate}

\exer{40}{Show that the two definitions are equivalent.}

\exer{30}{Write a third equivalent form of the definition using only the metric $d$.  (Do not mention open sets or $B_\eps(\cdot)$.)}

\defin Suppose $(X,d)$ is a metric space and that $\hat x = \lim_{n\to\infty} x_n$ for a sequence $(x_n)$.  We say that the sequence \textbf{converges}.

\exer{40}{Show that a sequence cannot converge to two different limits.}

Now consider complex-valued and real-valued functions.

\defin Suppose $(X,d)$ is a metric space and $f:X\to \CC$ is a function.  We say that $f$ is \textbf{continuous at $\hat x\in X$} if
\begin{enumerate}
\item for each $\eps>0$ there is $\delta>0$ so that $x \in B_\eps(\hat x)$ implies $|f(x)-f(\hat x)| < \eps$, or
\item for all sequences $(x_n)$ such that $\displaystyle \lim_{n\to\infty} x_n = \hat x$, it holds that $\displaystyle \lim_{n\to\infty} f(x_n) = f(\hat x)$.
\end{enumerate}

The second definition is often called ``sequential continuity''.

\defin We say that $f$ is \textbf{continuous} if it is continuous at every $\hat x\in X$.

\exer{40}{Let $f(x)=x^2$ for $x\in \RR$.  Show that $f$ is continuous.}

%\clearpage\newpage
\exer{40}{(This follows Example B on page \pageref{topic:metric}.  It is an extreme case.)  Let $X$ be any set and define the metric by $d(x,x)=0$ and $d(x,y)=1$ if $x\ne y$.  Let $f:X\to\CC$ be \emph{any} function.  Show that $f$ is continuous.}


\sect{Cauchy sequences and completeness}

When we have a topology on $X$ then we can discuss whether a given sequence $(x_n)$ of points in $X$ converges; see the above definition.  However, the definition requires that we have identified the limit $\hat x \in X$.  That is, convergence of a sequence means convergence to a particular limit.

In certain metric space topologies this becomes easier because there is an easier-to-check condition that is equivalent to convergence to some limit.  The condition only involves the distance between elements of the sequence.

\defin Suppose $(X,d)$ is a metric space and $(x_n)$ is a sequence of points in $X$.  We say that $(x_n)$ is a \textbf{Cauchy sequence} if for any $\eps>0$ there is $N$ so that if $m \ge N$ and $n\ge N$ then $d(x_m,x_n)<\eps$.

\exer{35}{Show that if a sequence $(x_n)$ converges to limit $\hat x$ then $(x_n)$ is Cauchy.}

\clearpage\newpage
\exer{50}{Rewrite the above definition for a normed vector space.  Compare the resulting definition to the one stated in section 2.1 of our textbook.  Show that the two definitions are the same.}

\defin Suppose $(X,d)$ is a metric space.  If every Cauchy sequence $(x_n)$ in $X$ has a limit, i.e.~there is $\hat x\in X$ so that $\displaystyle \lim_{n\to\infty} x_n=\hat x$, then we say that $(X,d)$ is \textbf{complete}.

We will accept the following Theorem without proof.

\thm The real numbers $\RR$, with the usual metric, is a complete metric space.  Likewise the complex numbers $\CC$ is a complete metric space.

\exer{60}{Let $x_0=2$ and define $x_{k+1}$ as the real zero of the tangent line to $y=x^2-2$ at $x_k$.  Argue geometrically, from the graph of $y=x^2-2$, that $(x_k)$ is Cauchy.  Note that $x_k$ is always a rational number.  What is the limit of $(x_k)$?}

With the above definition, we are now at the start of functional analysis!  The following is a major concept starting in section 2.1 of our textbook.

\defin Suppose $(\cV,\|\cdot\|)$ is a normed vector space, and that it is complete.  We call $(\cV,\|\cdot\|)$ a \textbf{Banach space}.


\clearpage\newpage
\sect{Compact sets}

Informally speaking, a compact subset in a metric space is well-approximated by finitely-many points.  The precise definition seems a little awkward when you first read it.

\defin Suppose $(X,d)$ is a metric space and $K \subset X$.  We say that $\{Y_\alpha\}$ is an \textbf{open cover} of $K$ if each $Y_\alpha$ is open set in $X$ and if
	$$K \subset \,\bigcup_{\alpha} \,Y_\alpha.$$

\defin We say $K$ is \textbf{compact} if from every open cover $\{Y_\alpha\}$ of $K$ we can choose a finite sub-collection $\{Y_{\alpha_j}\}_{j=1}^n$ which is also an open cover of $K$.

This definition is often said as ``every open cover of $K$ has a finite sub-cover''.

\exer{50}{For $X=\RR^2$ being the plane, sketch a subset $K$ and sketch the above definitions.}

\exer{40}{Suppose $K\subset X$ is a finite set.  Show that $K$ is compact.}

\exer{40}{Pretending to do numerical stuff: Sketch a grid of balls $B_\eps(x_{j,k})$ over the set $K=[0,1]^2$.  Make the balls an open cover.  This illustrates the informal description of ``compact.''}

\clearpage\newpage
\exer{50}{Suppose $K=[0,1] \subset \RR$.  Sketch the (direct) proof that $K$ is compact.  You may use the fact that $(\RR,|\cdot|)$ is a complete metric space.}

Closed and bounded sets in normed vector spaces are common objects of interest in functional analysis.  In finite dimensions these are compact sets.  We will see that they are \emph{not} generally compact in infinite dimensions (in the topology coming from the norm).

\thm (Heine-Borel theorem) If $K \subset \CC^n$ is closed and bounded then $K$ is compact.

A very important idea is that an optimization problem for a real-valued function can be solved if the function is continuous and the input set is compact.

\thm (Extreme value theorem) Suppose $(X,d)$ is a metric space, $K\subset X$, and $f:K\to\RR$ (or $f:X\to\RR$) is continuous.  If $K \subset X$ is compact then $f$ attains its maximum and minimum on $K$.

Precisely-stated, the conclusion of this theorem is that there are $p,q\in K$ so that
    $$f(p) \le f(x) \le f(q)$$
for all $x \in K$.



\sect{Linear map}

\sect{Linear independence}

\sect{Measure}

\sect{Integral}

\end{document}
