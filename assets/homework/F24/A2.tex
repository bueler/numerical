\documentclass[12pt]{amsart}
%prepared in AMSLaTeX, under LaTeX2e
\addtolength{\oddsidemargin}{-.6in} 
\addtolength{\evensidemargin}{-.6in}
\addtolength{\topmargin}{-.4in}
\addtolength{\textwidth}{1.2in}
\addtolength{\textheight}{.6in}

\renewcommand{\baselinestretch}{1.05}

\usepackage{verbatim,fancyvrb}

\usepackage{palatino}

\newtheorem*{thm}{Theorem}
\newtheorem*{defn}{Definition}
\newtheorem*{example}{Example}
\newtheorem*{problem}{Problem}
\newtheorem*{remark}{Remark}

\newcommand{\mtt}{\texttt}
\usepackage{alltt,xspace}
\newcommand{\mfile}[1]
{\medskip\begin{quote}\scriptsize \begin{alltt}\input{#1.m}\end{alltt} \normalsize\end{quote}\medskip}

\usepackage[final]{graphicx}
\newcommand{\mfigure}[1]{\includegraphics[height=2.5in,
width=3.5in]{#1.eps}}
\newcommand{\regfigure}[2]{\includegraphics[height=#2in,
keepaspectratio=true]{#1.eps}}
\newcommand{\widefigure}[3]{\includegraphics[height=#2in,
width=#3in]{#1.eps}}

\usepackage{amssymb}

\usepackage[pdftex, colorlinks=true, plainpages=false, linkcolor=black, citecolor=red, urlcolor=red]{hyperref}

% macros
\newcommand{\br}{\mathbf{r}}
\newcommand{\bv}{\mathbf{v}}
\newcommand{\bx}{\mathbf{x}}
\newcommand{\by}{\mathbf{y}}

\newcommand{\CC}{\mathbb{C}}
\newcommand{\RR}{\mathbb{R}}
\newcommand{\ZZ}{\mathbb{Z}}

\newcommand{\eps}{\epsilon}
\newcommand{\grad}{\nabla}
\newcommand{\lam}{\lambda}
\newcommand{\lap}{\triangle}

\newcommand{\ip}[2]{\ensuremath{\left<#1,#2\right>}}

%\renewcommand{\det}{\operatorname{det}}
\newcommand{\onull}{\operatorname{null}}
\newcommand{\rank}{\operatorname{rank}}
\newcommand{\range}{\operatorname{range}}

\newcommand{\Julia}{\textsc{Julia}\xspace}
\newcommand{\Matlab}{\textsc{Matlab}\xspace}
\newcommand{\Octave}{\textsc{Octave}\xspace}
\newcommand{\Python}{\textsc{Python}\xspace}

\newcommand{\prob}[1]{\bigskip\noindent\textbf{#1}\quad }

\newcommand{\chapexers}[2]{\prob{Chapter #1, pages #2, Exercises:}}
\newcommand{\exer}[2]{\prob{Exercise #1}}

\newcommand{\pts}[1]{(\emph{#1 pts}) }
\newcommand{\epart}[1]{\medskip\noindent\textbf{(#1)}\quad }
\newcommand{\ppart}[1]{\,\textbf{(#1)}\quad }

\newcommand*\circled[1]{\tikz[baseline=(char.base)]{
            \node[shape=ellipse,draw,inner sep=2pt] (char) {#1};}}


\begin{document}
\scriptsize \noindent Math 310 Numerical Analysis (Bueler) \hfill 30 August 2024 (version 1)
\normalsize

\medskip\bigskip

\Large\centerline{\textbf{Assignment \#2}}
\large
\bigskip

\centerline{\textbf{Due Friday, 13 September 2024, at the start of class}}
\bigskip
\normalsize

\thispagestyle{empty}

This Assignment is based on Chapter 4 of the textbook,\footnote{Greenbaum \& Chartier, \emph{Numerical Methods: Design, Analysis, and Computer Implementation of Algorithms}, Princeton University Press 2012).}  especially sections 4.1--4.4.  While bisection may be new to you, I claim you were introduced to Taylor's theorem and Newton's method in calculus.  Nonetheless, read these things carefully here.

\bigskip
\noindent Remember that when you turn in homework problems involving \Matlab (or language of your choice), the following two expectations will always apply:
\renewcommand{\labelenumi}{\arabic{enumi}.\,}
\begin{enumerate}
\item The commands that you used must be shown, along with the results.
\item Please strive to minimize use of paper: edit your result to remove extra space \emph{but} keep a clear distinction between your m-files, your input commands, and the computed results and/or figures.
\end{enumerate}

\bigskip\bigskip
\noindent \textbf{Do the following exercises from the textbook:}

\medskip
\renewcommand{\labelenumi}{{\footnotesize\underline{\textsc{Chapter \arabic{enumi}}}}}
\begin{enumerate}
\setcounter{enumi}{3}
\item ~
    \begin{itemize}
    \item Exercise 1 on page 102.  (\emph{Hint for (d): See section 4.4.})
    \item Exercise 2(a) on page 102.  (\emph{Start by plotting the function as requested.  Then write your code.  Please fully understand how bisection is programmed, even if you play with the book's code!})
    \item Exercise 3 on page 103.
    \item Exercise 4 on page 103.
    \item Exercise 6 on page 103.
    \item Exercise 7 on page 103.
    \end{itemize}
\end{enumerate}

\bigskip\bigskip
\noindent \textbf{Do the following additional problems:}

\bigskip
\noindent \textbf{P1.}  Consider applying bisection to $f(x)=x^2-2$, using an initial bracket $[1,2]$, to approximate $\sqrt{2}$ to within $10^{-2}$.  Calculate in advance how many steps are needed.  Now run the algorithm \emph{by hand} for that many steps, reporting each bracket.  (\emph{You won't even need a calculator.  In any case, don't run a \Matlab program.})




\end{document}
