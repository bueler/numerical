\documentclass[12pt]{amsart}
%prepared in AMSLaTeX, under LaTeX2e
\addtolength{\oddsidemargin}{-.6in} 
\addtolength{\evensidemargin}{-.6in}
\addtolength{\topmargin}{-.4in}
\addtolength{\textwidth}{1.2in}
\addtolength{\textheight}{.6in}

\renewcommand{\baselinestretch}{1.05}

\usepackage{verbatim,fancyvrb}

\usepackage{palatino}

\newtheorem*{thm}{Theorem}
\newtheorem*{defn}{Definition}
\newtheorem*{example}{Example}
\newtheorem*{problem}{Problem}
\newtheorem*{remark}{Remark}

\newcommand{\mtt}{\texttt}
\usepackage{alltt,xspace}
\newcommand{\mfile}[1]
{\medskip\begin{quote}\scriptsize \begin{alltt}\input{#1.m}\end{alltt} \normalsize\end{quote}\medskip}

\usepackage[final]{graphicx}
\newcommand{\mfigure}[1]{\includegraphics[height=2.5in,
width=3.5in]{#1.eps}}
\newcommand{\regfigure}[2]{\includegraphics[height=#2in,
keepaspectratio=true]{#1.eps}}
\newcommand{\widefigure}[3]{\includegraphics[height=#2in,
width=#3in]{#1.eps}}

\usepackage{amssymb}

\usepackage[pdftex, colorlinks=true, plainpages=false, linkcolor=black, citecolor=red, urlcolor=red]{hyperref}

% macros
\newcommand{\br}{\mathbf{r}}
\newcommand{\bv}{\mathbf{v}}
\newcommand{\bx}{\mathbf{x}}
\newcommand{\by}{\mathbf{y}}

\newcommand{\CC}{\mathbb{C}}
\newcommand{\RR}{\mathbb{R}}
\newcommand{\ZZ}{\mathbb{Z}}

\newcommand{\eps}{\epsilon}
\newcommand{\grad}{\nabla}
\newcommand{\lam}{\lambda}
\newcommand{\lap}{\triangle}

\newcommand{\ip}[2]{\ensuremath{\left<#1,#2\right>}}

%\renewcommand{\det}{\operatorname{det}}
\newcommand{\onull}{\operatorname{null}}
\newcommand{\rank}{\operatorname{rank}}
\newcommand{\range}{\operatorname{range}}

\newcommand{\Julia}{\textsc{Julia}\xspace}
\newcommand{\Matlab}{\textsc{Matlab}\xspace}
\newcommand{\Octave}{\textsc{Octave}\xspace}
\newcommand{\Python}{\textsc{Python}\xspace}

\newcommand{\prob}[1]{\bigskip\noindent\textbf{#1}\quad }

\newcommand{\chapexers}[2]{\prob{Chapter #1, pages #2, Exercises:}}
\newcommand{\exer}[2]{\prob{Exercise #1}}

\newcommand{\pts}[1]{(\emph{#1 pts}) }
\newcommand{\epart}[1]{\medskip\noindent\textbf{(#1)}\quad }
\newcommand{\ppart}[1]{\,\textbf{(#1)}\quad }

\newcommand*\circled[1]{\tikz[baseline=(char.base)]{
            \node[shape=ellipse,draw,inner sep=2pt] (char) {#1};}}


\begin{document}
\scriptsize \noindent Math 310 Numerical Analysis (Bueler) \hfill 26 August 2024 (version 2)
\normalsize

\medskip\bigskip

\Large\centerline{\textbf{Assignment \#1}}
\large
\bigskip

\centerline{\textbf{Due Wednesday, 4 September 2024, at the start of class}}
\bigskip
\normalsize

\thispagestyle{empty}

\bigskip
\noindent Make sure you have a copy of the textbook:

\begin{quote}
Greenbaum \& Chartier, \emph{Numerical Methods: Design, Analysis, and Computer Implementation of Algorithms}, Princeton University Press 2012.
\end{quote}

\bigskip
\noindent Please lightly read Chapter 1; there will be no exercises on this Chapter.

\bigskip
\noindent Please read Chapter 2 in detail.  You will need to find \Matlab online, or download a copy, or choose and start another language.  The main purpose of Chapter 2 and this Assignment is to familiarize you with \Matlab (equivalently, \Octave).  Even if you use another language for the rest of the semester, it is probably easiest to do the problems here in \Matlab or \Octave.  (Modifications for \Julia are straightforward.)

\bigskip
\noindent If you are new to \Matlab, it would be smart to \emph{input and check each \Matlab example done in Chapter 2}, as well as doing the Exercises below.  Make sure you can create a new M-file (script), save it, edit it, and run it at the command line by typing its name.  Similarly for function M-files.

\bigskip
\noindent When you turn in Assignments, two expectations always apply:
\renewcommand{\labelenumi}{\arabic{enumi}.\,}
\begin{enumerate}
\item The \emph{commands that you used must be shown}, along with the results.
\item Turn in a non-wasteful and readable single document.  In particular, edit your result to remove excess space, \emph{but} keep a clear distinction between your by-hand calculations, programs, input commands, and computed results.  My solutions will be good style examples.
\end{enumerate}

\bigskip\bigskip
\noindent \textbf{Do the following exercises:}

\medskip
\renewcommand{\labelenumi}{{\footnotesize\underline{\textsc{Chapter \arabic{enumi}}}}}
\begin{enumerate}
\setcounter{enumi}{1}
\item ~
    \begin{itemize}
    \item Exercise 2 on page 32.
    \item Exercise 3 on page 32.
    \item Exercise 4 on page 32. (\emph{The table should be neat, have three columns, and take about 14 lines only.})
    \item Exercise 6 (a), (c) on page 34.
    \item Exercise 9 on page 35.
    \item Extra Credit.  Can you do Exercise 3 on page 32 (above) in less than 80 total characters, including spaces and separators?
    \end{itemize}
\end{enumerate}

\end{document}
