\documentclass[12pt]{amsart}
%prepared in AMSLaTeX, under LaTeX2e
\addtolength{\oddsidemargin}{-.65in} 
\addtolength{\evensidemargin}{-.65in}
\addtolength{\topmargin}{-.3in}
\addtolength{\textwidth}{1.1in}
\addtolength{\textheight}{.3in}

\renewcommand{\baselinestretch}{1.05}

\usepackage{verbatim,fancyvrb}

\usepackage{palatino}

\usepackage[dvipsnames]{xcolor}

\newtheorem*{thm}{Theorem}
\newtheorem*{defn}{Definition}
\newtheorem*{example}{Example}
\newtheorem*{problem}{Problem}
\newtheorem*{remark}{Remark}

\newcommand{\mtt}{\texttt}
\usepackage{alltt,xspace}
\newcommand{\mfile}[1]
{\medskip\begin{quote}\scriptsize \begin{alltt}\input{#1.m}\end{alltt} \normalsize\end{quote}\medskip}

\usepackage[final]{graphicx}
\newcommand{\mfigure}[1]{\includegraphics[height=2.5in,
width=3.5in]{#1.eps}}
\newcommand{\regfigure}[2]{\includegraphics[height=#2in,
keepaspectratio=true]{#1.eps}}
\newcommand{\widefigure}[3]{\includegraphics[height=#2in,
width=#3in]{#1.eps}}

\usepackage{amssymb}

\usepackage[pdftex, colorlinks=true, plainpages=false, linkcolor=black, citecolor=red, urlcolor=red]{hyperref}

% macros
\newcommand{\bb}{\mathbf{b}}
\newcommand{\br}{\mathbf{r}}
\newcommand{\bv}{\mathbf{v}}
\newcommand{\bx}{\mathbf{x}}
\newcommand{\by}{\mathbf{y}}

\newcommand{\CC}{\mathbb{C}}
\newcommand{\RR}{\mathbb{R}}
\newcommand{\ZZ}{\mathbb{Z}}

\newcommand{\eps}{\epsilon}
\newcommand{\grad}{\nabla}
\newcommand{\lam}{\lambda}
\newcommand{\lap}{\triangle}

\newcommand{\ip}[2]{\ensuremath{\left<#1,#2\right>}}

%\renewcommand{\det}{\operatorname{det}}
\newcommand{\onull}{\operatorname{null}}
\newcommand{\rank}{\operatorname{rank}}
\newcommand{\range}{\operatorname{range}}

\newcommand{\Julia}{\textsc{Julia}\xspace}
\newcommand{\Matlab}{\textsc{Matlab}\xspace}
\newcommand{\Octave}{\textsc{Octave}\xspace}
\newcommand{\Python}{\textsc{Python}\xspace}

\newcommand{\prob}[1]{\bigskip\noindent\textbf{#1.}\quad }

\newcommand{\pts}[1]{(\emph{#1 pts}) }
\newcommand{\epart}[1]{\medskip\noindent\textbf{(#1)}\quad }
\newcommand{\ppart}[1]{\,\textbf{(#1)}\quad }

\newcommand*\circled[1]{\tikz[baseline=(char.base)]{
            \node[shape=ellipse,draw,inner sep=2pt] (char) {#1};}}


\begin{document}
\scriptsize \noindent Math 426 Numerical Analysis (Bueler) \hfill 15 October 2024
\normalsize

\medskip\bigskip

\Large\centerline{\textbf{Assignment \#6}}
\large
\bigskip

\centerline{\textbf{Due Friday, 25 October 2024, at the start of class}}
\medskip
\normalsize

\thispagestyle{empty}

\begin{quote}
{\small
This Assignment is based on Chapter 8 of the textbook.\footnote{Greenbaum \& Chartier, \emph{Numerical Methods: Design, Analysis, and Computer Implementation of Algorithms}, Princeton University Press 2012).} Please read sections 8.1, 8.2, 8.4, 8.5, and 8.6.  You can ignore the material in sections 8.3 and 8.7.  You can also ignore the material on barycentric form in subsection 8.2, and the material on \texttt{chebfun} in section 8.5, and subsection 8.6.1.

\medskip
\noindent When you turn in homework problems, please put the problems in the order they appear below.  Also, two expectations always apply:
\renewcommand{\labelenumi}{\arabic{enumi}.\,}
\begin{enumerate}
\item When you use \Matlab, or other language of your choice, the commands you used must be shown, along with the results.
\item Please minimize use of paper; edit your result to remove extra space.  However, please keep a clear distinction between codes, input commands, and computed results and/or figures.
\end{enumerate}
}

\medskip
\noindent Finally, especially relevant to this Assignment, when generating the graph of a smooth function, please use a fine grid.  This technique has been shown in class, and several times in my solutions.
\end{quote}

\bigskip
\noindent \textbf{Do the following exercises from the textbook:}

\medskip
\renewcommand{\labelenumi}{{\footnotesize\underline{\textsc{Chapter \arabic{enumi}}}}}
\begin{enumerate}
\setcounter{enumi}{7}
\item ~
    \begin{itemize}
    \item Exercise 1 (a) and (b) on page 206--207.
    \item Exercise 2 on page 207.
    \item Exercise 4 (a) and (b) on pages 207--208.
    \item Exercise 5 on page 208.
    \item Exercise 7 (a) and (c) on pages 208--209.
    \end{itemize}
\end{enumerate}

\bigskip
\noindent \textbf{Do the following additional problems:}

\prob{P6}  \ppart{a}  Let $f(x)=\arctan(x)$ and suppose $x_0=-\frac{1}{\sqrt{3}},x_1=0,x_2=\frac{1}{\sqrt{3}},x_3=1$.  Find the Vandermonde and Lagrange forms of the interpolating polynomial $p(x)$, i.e.~so that $p(x_i)=y_i$.  You may use a computer to solve the linear systems that arise for the Vandermonde form.  By simplifying, check that these are the same polynomial.

\epart{b}  Use the polynomial interpolation error theorem, namely Theorem 8.4.1, to estimate $|f(x)-p(x)|$ on the interval $[-1,1]$.  For this job you will need a formula for a higher derivative of $f$, and an estimate of the maximum of this derivative on the interval $[-1,1]$.  You may use tools for these tasks, e.g.~Wolfram Alpha, but please make it clear how you do that.

\prob{P7}  \ppart{a}  As stated on page 193 of the text, it is known that the Chebyshev points $x_j=\cos(\pi j/n)$, for $j=0,1,\dots,n$, satisfy
        $$\max_{-1\le x\le 1} \left|(x-x_0)(x-x_1)\cdots(x-x_n)\right| \le \frac{1}{2^{n-1}}.$$
    Create plots for $n=8$ and $n=20$ which illustrate this fact.
    % n=20;  x = cos(pi*(0:n)/n);  xx = -1:.001:1;  g = xx-x(1);  for j=2:n+1,  g = g .* (xx-x(j)); end,  plot(xx,abs(g),xx,2^(1-n)*ones(size(xx))),  grid on
    
\epart{b}  Consider $f(x) = \cos(4x+1)$ on the interval $[-1,1]$.  Use the fact stated in (i), and Theorem 8.4.1, to find $n$ so that the interpolating polynomial $p(x)$ using the Chebyshev points satisfies $|f(x)-p(x)|<10^{-14}$.  (\emph{You are \emph{not} asked to find $p$ itself!})

\end{document}
