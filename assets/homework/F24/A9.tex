\documentclass[12pt]{amsart}
%prepared in AMSLaTeX, under LaTeX2e
\addtolength{\oddsidemargin}{-.65in} 
\addtolength{\evensidemargin}{-.65in}
\addtolength{\topmargin}{-.3in}
\addtolength{\textwidth}{1.1in}
\addtolength{\textheight}{.3in}

\renewcommand{\baselinestretch}{1.05}

\usepackage{verbatim,fancyvrb}

\usepackage{palatino}

\usepackage[dvipsnames]{xcolor}

\newtheorem*{thm}{Theorem}
\newtheorem*{defn}{Definition}
\newtheorem*{example}{Example}
\newtheorem*{problem}{Problem}
\newtheorem*{remark}{Remark}

\newcommand{\mtt}{\texttt}
\usepackage{alltt,xspace}
\newcommand{\mfile}[1]
{\medskip\begin{quote}\scriptsize \begin{alltt}\input{#1.m}\end{alltt} \normalsize\end{quote}\medskip}

\usepackage[final]{graphicx}
\newcommand{\mfigure}[1]{\includegraphics[height=2.5in,
width=3.5in]{#1.eps}}
\newcommand{\regfigure}[2]{\includegraphics[height=#2in,
keepaspectratio=true]{#1.eps}}
\newcommand{\widefigure}[3]{\includegraphics[height=#2in,
width=#3in]{#1.eps}}

\usepackage{amssymb}

\usepackage[pdftex, colorlinks=true, plainpages=false, linkcolor=black, citecolor=red, urlcolor=red]{hyperref}

% macros
\newcommand{\bb}{\mathbf{b}}
\newcommand{\br}{\mathbf{r}}
\newcommand{\bv}{\mathbf{v}}
\newcommand{\bx}{\mathbf{x}}
\newcommand{\by}{\mathbf{y}}

\newcommand{\CC}{\mathbb{C}}
\newcommand{\RR}{\mathbb{R}}
\newcommand{\ZZ}{\mathbb{Z}}

\newcommand{\eps}{\epsilon}
\newcommand{\grad}{\nabla}
\newcommand{\lam}{\lambda}
\newcommand{\lap}{\triangle}

\newcommand{\ip}[2]{\ensuremath{\left<#1,#2\right>}}

%\renewcommand{\det}{\operatorname{det}}
\newcommand{\onull}{\operatorname{null}}
\newcommand{\rank}{\operatorname{rank}}
\newcommand{\range}{\operatorname{range}}

\newcommand{\Julia}{\textsc{Julia}\xspace}
\newcommand{\Matlab}{\textsc{Matlab}\xspace}
\newcommand{\Octave}{\textsc{Octave}\xspace}
\newcommand{\Python}{\textsc{Python}\xspace}

\newcommand{\prob}[1]{\bigskip\noindent\textbf{#1.}\quad }

\newcommand{\pts}[1]{(\emph{#1 pts}) }
\newcommand{\epart}[1]{\medskip\noindent\textbf{(#1)}\quad }
\newcommand{\ppart}[1]{\,\textbf{(#1)}\quad }

\newcommand*\circled[1]{\tikz[baseline=(char.base)]{
            \node[shape=ellipse,draw,inner sep=2pt] (char) {#1};}}


\begin{document}
\scriptsize \noindent Math 426 Numerical Analysis (Bueler) \hfill 18 November 2024
\normalsize

\medskip\bigskip

\Large\centerline{\textbf{Assignment \#8}}
\large
\bigskip

\centerline{\textbf{Due Monday, 2 December 2024, at the start of class}}
\medskip
\normalsize

\thispagestyle{empty}

\begin{quote}
{\small
This Assignment is based on Chapter 11 of our textbook.\footnote{Greenbaum \& Chartier, \emph{Numerical Methods: Design, Analysis, and Computer Implementation of Algorithms}, Princeton University Press 2012).}  Please read all of sections 11.1 and 11.2; the latter is very substantial!  You can skip section 11.3, but please read section 11.4 through page 289; you can skip the rest (pages 290--294).

\medskip
\noindent These expectations always apply to homework:
\renewcommand{\labelenumi}{\arabic{enumi}.\,}
\begin{enumerate}
\item Please put the problems in the order they appear below.
\item When you use \Matlab/etc, show the commands and the results.
\item Keep a clear distinction between codes, input commands, computed results, and figures.
\item Other than the text you write, please minimize use of paper.
\end{enumerate}
}
\end{quote}

\medskip
\noindent \textbf{Do these problems:}


\prob{P13} FIXME Lorenz attractor using \texttt{ode45}


\prob{P14} FIXME Exercise 14 in Chapter 11, but shortened; this will be an in-class activity


\end{document}
