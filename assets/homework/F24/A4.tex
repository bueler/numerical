\documentclass[12pt]{amsart}
%prepared in AMSLaTeX, under LaTeX2e
\addtolength{\oddsidemargin}{-.6in} 
\addtolength{\evensidemargin}{-.6in}
\addtolength{\topmargin}{-.4in}
\addtolength{\textwidth}{1.2in}
\addtolength{\textheight}{.6in}

\renewcommand{\baselinestretch}{1.05}

\usepackage{verbatim,fancyvrb}

\usepackage{palatino}

\newtheorem*{thm}{Theorem}
\newtheorem*{defn}{Definition}
\newtheorem*{example}{Example}
\newtheorem*{problem}{Problem}
\newtheorem*{remark}{Remark}

\newcommand{\mtt}{\texttt}
\usepackage{alltt,xspace}
\newcommand{\mfile}[1]
{\medskip\begin{quote}\scriptsize \begin{alltt}\input{#1.m}\end{alltt} \normalsize\end{quote}\medskip}

\usepackage[final]{graphicx}
\newcommand{\mfigure}[1]{\includegraphics[height=2.5in,
width=3.5in]{#1.eps}}
\newcommand{\regfigure}[2]{\includegraphics[height=#2in,
keepaspectratio=true]{#1.eps}}
\newcommand{\widefigure}[3]{\includegraphics[height=#2in,
width=#3in]{#1.eps}}

\usepackage{amssymb}

\usepackage[pdftex, colorlinks=true, plainpages=false, linkcolor=black, citecolor=red, urlcolor=red]{hyperref}

% macros
\newcommand{\br}{\mathbf{r}}
\newcommand{\bv}{\mathbf{v}}
\newcommand{\bx}{\mathbf{x}}
\newcommand{\by}{\mathbf{y}}

\newcommand{\CC}{\mathbb{C}}
\newcommand{\RR}{\mathbb{R}}
\newcommand{\ZZ}{\mathbb{Z}}

\newcommand{\eps}{\epsilon}
\newcommand{\grad}{\nabla}
\newcommand{\lam}{\lambda}
\newcommand{\lap}{\triangle}

\newcommand{\ip}[2]{\ensuremath{\left<#1,#2\right>}}

%\renewcommand{\det}{\operatorname{det}}
\newcommand{\onull}{\operatorname{null}}
\newcommand{\rank}{\operatorname{rank}}
\newcommand{\range}{\operatorname{range}}

\newcommand{\Julia}{\textsc{Julia}\xspace}
\newcommand{\Matlab}{\textsc{Matlab}\xspace}
\newcommand{\Octave}{\textsc{Octave}\xspace}
\newcommand{\Python}{\textsc{Python}\xspace}

\newcommand{\prob}[1]{\bigskip\noindent\textbf{#1}\quad }

\newcommand{\chapexers}[2]{\prob{Chapter #1, pages #2, Exercises:}}
\newcommand{\exer}[2]{\prob{Exercise #1}}

\newcommand{\pts}[1]{(\emph{#1 pts}) }
\newcommand{\epart}[1]{\medskip\noindent\textbf{(#1)}\quad }
\newcommand{\ppart}[1]{\,\textbf{(#1)}\quad }

\newcommand*\circled[1]{\tikz[baseline=(char.base)]{
            \node[shape=ellipse,draw,inner sep=2pt] (char) {#1};}}


\begin{document}
\scriptsize \noindent Math 426 Numerical Analysis (Bueler) \hfill 20 September 2024
\normalsize

\medskip\bigskip

\Large\centerline{\textbf{Assignment \#4}}
\large
\bigskip

\centerline{\textbf{Due Monday, 30 September 2024, at the start of class}}
\bigskip
\normalsize

\thispagestyle{empty}

This Assignment is based on Chapters 4 and 5 of the textbook,\footnote{Greenbaum \& Chartier, \emph{Numerical Methods: Design, Analysis, and Computer Implementation of Algorithms}, Princeton University Press 2012).}  Please read these Chapters!  When you turn in homework problems, please put the problems in the order they appear below.  Also, the following two expectations will always apply:
\renewcommand{\labelenumi}{\arabic{enumi}.\,}
\begin{enumerate}
\item If you use \Matlab, or the language of your choice, then the commands that you used must be shown, along with the results.
\item Please strive to minimize use of paper: edit your result to remove extra space \emph{but} keep a clear distinction between your m-files, your input commands, and the computed results and/or figures.
\end{enumerate}

\bigskip\bigskip
\noindent \textbf{Do the following exercises from the textbook:}

\medskip
\renewcommand{\labelenumi}{{\footnotesize\underline{\textsc{Chapter \arabic{enumi}}}}}
\begin{enumerate}
\setcounter{enumi}{3}
\item ~
    \begin{itemize}
    \item Exercise 12 on page 104.
    \item Exercise 14 on page 104.
    \item Exercise 15 on page 104.
    \end{itemize}
\item ~
    \begin{itemize}
    \item Exercise 3 on page 120.
    \item Exercise 8 on page 120.
    \end{itemize}
\end{enumerate}

\bigskip\bigskip
\noindent \textbf{Do the following additional problems:}

\bigskip
\noindent \textbf{P2.}  By plotting, check that the graphs $y=e^{-x^2}$ and $y=x^2$ cross between $x=0.7$ and $x=0.8$.  Use the secant method, and \Matlab or language of your choice, to find the crossing point to within $10^{-14}$ using $x_0=0.7$ and $x_1=0.75$.  (\emph{You may write a script, or work at the command line, but show sufficient inputs/outputs to illustrate your understanding.})  Also use Newton's method with $x_0=0.75$ to do the same job.  How many iterations of the secant method are needed?  Of Newton's method?

\bigskip
\noindent \textbf{P3.}  Consider the problem of solving $f(x)=x^3-7x+2=0$ on the interval $[0,1]$.  Confirm, by using the intermediate value theorem, that there is a solution on that interval.  Write down Newton's method for this problem as a fixed-point iteration $x_{k+1} = \varphi(x_k)$.  What is $\varphi(x)$ in this case?  By computing $\varphi'(x)$, and plotting it on the interval $[0,1]$, find an interval $I$ so that if $x_0$ is in $I$ then you can be sure that Newton's method will converge.  (\emph{Use Theorem 4.5.1 to do the last part.})




\end{document}
