\documentclass[12pt]{amsart}
%prepared in AMSLaTeX, under LaTeX2e
\addtolength{\oddsidemargin}{-.65in} 
\addtolength{\evensidemargin}{-.65in}
\addtolength{\topmargin}{-.3in}
\addtolength{\textwidth}{1.1in}
\addtolength{\textheight}{.3in}

\renewcommand{\baselinestretch}{1.05}

\usepackage{verbatim,fancyvrb}

\usepackage{palatino}

\usepackage[dvipsnames]{xcolor}

\newtheorem*{thm}{Theorem}
\newtheorem*{defn}{Definition}
\newtheorem*{example}{Example}
\newtheorem*{problem}{Problem}
\newtheorem*{remark}{Remark}

\newcommand{\mtt}{\texttt}
\usepackage{alltt,xspace}
\newcommand{\mfile}[1]
{\medskip\begin{quote}\scriptsize \begin{alltt}\input{#1.m}\end{alltt} \normalsize\end{quote}\medskip}

\usepackage[final]{graphicx}
\newcommand{\mfigure}[1]{\includegraphics[height=2.5in,
width=3.5in]{#1.eps}}
\newcommand{\regfigure}[2]{\includegraphics[height=#2in,
keepaspectratio=true]{#1.eps}}
\newcommand{\widefigure}[3]{\includegraphics[height=#2in,
width=#3in]{#1.eps}}

\usepackage{amssymb}

\usepackage[pdftex, colorlinks=true, plainpages=false, linkcolor=black, citecolor=red, urlcolor=red]{hyperref}

% macros
\newcommand{\bb}{\mathbf{b}}
\newcommand{\br}{\mathbf{r}}
\newcommand{\bv}{\mathbf{v}}
\newcommand{\bx}{\mathbf{x}}
\newcommand{\by}{\mathbf{y}}

\newcommand{\CC}{\mathbb{C}}
\newcommand{\RR}{\mathbb{R}}
\newcommand{\ZZ}{\mathbb{Z}}

\newcommand{\eps}{\epsilon}
\newcommand{\grad}{\nabla}
\newcommand{\lam}{\lambda}
\newcommand{\lap}{\triangle}

\newcommand{\ip}[2]{\ensuremath{\left<#1,#2\right>}}

%\renewcommand{\det}{\operatorname{det}}
\newcommand{\onull}{\operatorname{null}}
\newcommand{\rank}{\operatorname{rank}}
\newcommand{\range}{\operatorname{range}}

\newcommand{\Julia}{\textsc{Julia}\xspace}
\newcommand{\Matlab}{\textsc{Matlab}\xspace}
\newcommand{\Octave}{\textsc{Octave}\xspace}
\newcommand{\Python}{\textsc{Python}\xspace}

\newcommand{\prob}[1]{\bigskip\noindent\textbf{#1.}\quad }

\newcommand{\pts}[1]{(\emph{#1 pts}) }
\newcommand{\epart}[1]{\medskip\noindent\textbf{(#1)}\quad }
\newcommand{\ppart}[1]{\,\textbf{(#1)}\quad }

\newcommand*\circled[1]{\tikz[baseline=(char.base)]{
            \node[shape=ellipse,draw,inner sep=2pt] (char) {#1};}}


\begin{document}
\scriptsize \noindent Math 426 Numerical Analysis (Bueler) \hfill 28 October 2024
\normalsize

\medskip\bigskip

\Large\centerline{\textbf{Assignment \#7}}
\large
\bigskip

\centerline{\textbf{Due Friday, 8 November 2024, at the start of class}}
\medskip
\normalsize

\thispagestyle{empty}

\begin{quote}
{\small
This Assignment is based mostly on Chapter 10 of the textbook.\footnote{Greenbaum \& Chartier, \emph{Numerical Methods: Design, Analysis, and Computer Implementation of Algorithms}, Princeton University Press 2012).}  Please read sections 10.1, 10.2, and FIXME

\medskip
\noindent These expectations always apply to homework:
\renewcommand{\labelenumi}{\arabic{enumi}.\,}
\begin{enumerate}
\item Please put the problems in the order they appear below.
\item When you use \Matlab/etc.~the commands you used must be shown, along with the results.
\item Please keep a clear distinction between codes, input commands, and computed results and/or figures.
\item Other than the text you write, please minimize use of paper.  For example, computer outputs and figures do not need extra space.
\end{enumerate}
}
\end{quote}

\bigskip
\noindent \textbf{Do the following exercises from the textbook:}

\medskip
\renewcommand{\labelenumi}{{\footnotesize\underline{\textsc{Chapter \arabic{enumi}}}}}
\begin{enumerate}
\setcounter{enumi}{7}
\item ~
    \begin{itemize}
    \item Exercise FIXME chapters 8 and 10
    \end{itemize}
\end{enumerate}

\bigskip
\noindent \textbf{Do the following additional problems:}

\prob{P9}  \ppart{a} clenshaw curtis

\epart{b}  

\prob{P8}  Find some grid paper with roughly 1/4 inch grid and trace the outline of your hand on it.  (\emph{I googled ``printable grid paper,'' etc.})  Add 30 to 50 \emph{roughly} equally-spaced points along the outline, generally including tips of fingers and saddle points between fingers.  (\emph{At this point my result looked like the figure below, with $n=36$ points.  You can read values off this graph if you want, and you'll get a picture of \emph{my} hand, but yours is more fun!})  Type into the Matlab (or other) editor, so you only have to do it once, the $(x_k,y_k)$ locations of each point, for $k=1,\dots,n$, choosing coordinates on the grid paper in some manner.

Now the idea is to get a cubic spline interpolant which is a \textbf{parameterized curve} $(x(t),y(t))$.  The indexing can be regarded as $t$-values, namely $t_k=k$ for $k=1,\dots,n$.  The function $x(t)$ interpolates all the pairs $(t_k,x_k)$ and $y(t)$ interpolates all the $(t_k,y_k)$ pairs.

Plot the interpolating parameterized curve $(x(t),y(t))$ in the $x,y$ plane using the Matlab \texttt{interp1()} function (twice), with the last argument \texttt{'spline'}.  For plotting you will need to generate a fine grid of $t$ values on the interval $[1,n]$.  Turn in the plot and your code.

Only plot the $(x,y)$ values in the main figure, but feel free to generate separate figures for the functions $x(t)$ and $y(t)$; this is optional.  Other than the data for the points $(x_k,y_k)$, your Matlab program should only be a few lines.

\end{document}
