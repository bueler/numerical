\documentclass[12pt]{amsart}
%prepared in AMSLaTeX, under LaTeX2e
\addtolength{\oddsidemargin}{-.55in} 
\addtolength{\evensidemargin}{-.55in}
\addtolength{\topmargin}{-.4in}
\addtolength{\textwidth}{1.1in}
\addtolength{\textheight}{.6in}

\renewcommand{\baselinestretch}{1.05}

\usepackage{verbatim} % for "comment" environment

\usepackage{palatino}

\newtheorem*{thm}{Theorem}
\newtheorem*{defn}{Definition}
\newtheorem*{example}{Example}
\newtheorem*{problem}{Problem}
\newtheorem*{remark}{Remark}

\usepackage{fancyvrb,xspace,dsfont}

\usepackage[final]{graphicx}

% macros
\usepackage{amssymb}

\usepackage{hyperref}
\hypersetup{pdfauthor={Ed Bueler},
            pdfcreator={pdflatex},
            colorlinks=true,
            citecolor=blue,
            linkcolor=red,
            urlcolor=blue,
            }

\newcommand{\br}{\mathbf{r}}
\newcommand{\bv}{\mathbf{v}}
\newcommand{\bx}{\mathbf{x}}
\newcommand{\by}{\mathbf{y}}

\newcommand{\cD}{\mathcal{D}}
\newcommand{\cF}{\mathcal{F}}
\newcommand{\cH}{\mathcal{H}}
\newcommand{\cL}{\mathcal{L}}
\newcommand{\cV}{\mathcal{V}}
\newcommand{\cW}{\mathcal{W}}

\newcommand{\CC}{\mathbb{C}}
\newcommand{\NN}{\mathbb{N}}
\newcommand{\RR}{\mathbb{R}}
\newcommand{\ZZ}{\mathbb{Z}}

\newcommand{\eps}{\epsilon}
\newcommand{\grad}{\nabla}
\newcommand{\lam}{\lambda}
\newcommand{\lap}{\triangle}

\newcommand{\ip}[2]{\ensuremath{\left<#1,#2\right>}}

\newcommand{\image}{\operatorname{im}}
\newcommand{\onull}{\operatorname{null}}
\newcommand{\rank}{\operatorname{rank}}
\newcommand{\range}{\operatorname{range}}
\newcommand{\trace}{\operatorname{tr}}

\newcommand{\Span}{\operatorname{span}}

\newcommand{\prob}[1]{\bigskip\noindent\textbf{#1.}\quad }
\newcommand{\exer}[2]{\prob{Exercise #2 in Lecture #1}}

\newcommand{\pts}[1]{(\emph{#1 pts}) }
\newcommand{\epart}[1]{\medskip\noindent\textbf{(#1)}\quad }
\newcommand{\ppart}[1]{\,\textbf{(#1)}\quad }

\newcommand{\Matlab}{\textsc{Matlab}\xspace}
\newcommand{\Octave}{\textsc{Octave}\xspace}
\newcommand{\Python}{\textsc{Python}\xspace}
\newcommand{\Julia}{\textsc{Julia}\xspace}

\newcommand{\fl}{\operatorname{fl}}

\newcommand{\ds}{\displaystyle}

\DefineVerbatimEnvironment{mVerb}{Verbatim}{numbersep=2mm,
frame=lines,framerule=0.1mm,framesep=2mm,xleftmargin=4mm,fontsize=\footnotesize}

\newcommand{\nex}{\medskip\noindent}


\begin{document}
\scriptsize \noindent Math 617 Functional Analysis (Bueler) \hfill \emph{version 3; assigned 29 March 2024}
\normalsize\medskip

\Large\centerline{\textbf{Assignment 6}}
\large
\medskip

\centerline{\textbf{Due Wednesday 10 April 2024}}
\medskip
\normalsize

\thispagestyle{empty}

\bigskip
\noindent This Assignment is based primarily on sections 3.1, 3.2, 3.3, 3.4, and 4.1 of our textbook, Borthwick (2020)~\emph{Spectral Theory: Basic Concepts and Applications}, Springer.  Note that 3.5 is skipped for now.

\medskip
\noindent \textsc{Please do the following exercises.}
\smallskip

\renewcommand{\SS}{\mathbb{S}}

\prob{P28}  \emph{Hint.  You did Exercise 2.5, and you may use its result.}

\medskip\noindent Show that if $T\in\cL(\cH)$ then $\|T\|=\|T^*\|$ and $\|T^*T\|=\|T\|^2$.


\prob{P29}  \emph{Recall that $U\in\cL(\cH)$ is \emph{unitary} if it is bijective (one-to-one and onto) and an isometry.  Note that we will be able to show that the spectrum of $U=M_f$ in parts \emph{\textbf{(c)}} and \emph{\textbf{(d)}} is the entire unit circle $\SS$.  Regarding part \emph{\textbf{(d)}}, Section 4.3 defines the phrase ``approximate eigenvalue'' for this situation, where $Uv_n \approx zv_n$.}

\epart{a}  Note $\cH$ is a complex Hilbert space.  Use the correct\footnote{See Assignment \# 3 for the correct form which you proved.  The textbook's version, (2.17) on page 17, has sign errors.} polarization identity to show that if $U$ is unitary then $\ip{Ux}{Uy}=\ip{x}{y}$ for all $x,y$ in $\cH$.

\epart{b}  Suppose $\lambda \in\CC$ is an eigenvalue of a unitary operator $U$.  Show that $\lambda$ is on the unit circle $\SS = \{z\in\CC\,:\,|z|=1\}$.

\epart{c}  Let $\cH = L^2(0,1)$ and fix the function $f(x) = e^{2\pi i x}$.  Show that the multiplication operator $U=M_f$ is bounded and unitary.  (\emph{Hint. Bounded is easy; use Example 2.8 from the textbook. Unitary is also easy.})  Show that this $U$ has no eigenvalues.

\epart{d}  For $U$ from part \textbf{(c)}, and each $z\in\SS$, find a sequence of functions $v_n\in\cH$ so that $\|v_n\|=1$ and $\ds \lim_{n\to\infty} \|(U-z)v_n\|=0$.


\prob{P30} \emph{I am deliberately suppressing most details of \emph{closed} operators (section 3.3).  However, this exercise uses Definition 3.8, and suggests what ``closed'' is saying.}

\epart{a} Let $T$ be an operator on $\cH$, so $\cD(T)$ is a dense subspace.  Show that
	$$\|u\|_T = \left(\|u\|^2 + \|Tu\|^2\right)^{1/2}$$
defines a norm on $\cD(T)$, called the \emph{graph norm} for $T$.

\epart{b} Show that $T$ is closed as an operator on $\cH$ if and only if the normed vector space $(\cD(T),\|\cdot\|_T)$ is complete.


\clearpage\newpage
\prob{P31} \emph{Page 47 of the text is important as it defines \emph{self-adjoint}, \emph{symmetric}, and \emph{positive} for unbounded operators.  Hint. Consider vectors of the form $u+v$ and $u-iv$ for $u,v\in\cD(A)$.  Also, if a number is nonnegative then it is real!}

\medskip\noindent Prove that a positive operator $A$ on a complex Hilbert space $\cH$ is symmetric.


\prob{P32}  \emph{Hint.  You will need the fact that $v' \in L^2(0,1)$ implies $v' \in L^1(0,1)$ implies $v \in C[0,1]$ implies $v(0)$ is well-defined.  And, of course, integration by parts.}

\medskip\noindent Let $\cH=L^2(0,1)$.  Consider $T=d/dx$, an unbounded operator on $\cH$, with domain $\cD(T) = \{u\in C^1[0,1]\,:\,u(1)=0\}$.  Show that
    $$\cD(T^*) = \left\{v\in AC[0,1]\,:\,v'\in\cH \text{ and } v(0)=0\right\}$$
and that $T^*=-d/dx$.

\end{document}
