\documentclass[12pt]{amsart}
%prepared in AMSLaTeX, under LaTeX2e
\addtolength{\oddsidemargin}{-.55in} 
\addtolength{\evensidemargin}{-.55in}
\addtolength{\topmargin}{-.4in}
\addtolength{\textwidth}{1.1in}
\addtolength{\textheight}{.6in}

\renewcommand{\baselinestretch}{1.05}

\usepackage{verbatim} % for "comment" environment

\usepackage{palatino}

\newtheorem*{thm}{Theorem}
\newtheorem*{defn}{Definition}
\newtheorem*{example}{Example}
\newtheorem*{problem}{Problem}
\newtheorem*{remark}{Remark}

\usepackage{fancyvrb,xspace,dsfont}

\usepackage[final]{graphicx}

% macros
\usepackage{amssymb}

\usepackage{hyperref}
\hypersetup{pdfauthor={Ed Bueler},
            pdfcreator={pdflatex},
            colorlinks=true,
            citecolor=blue,
            linkcolor=red,
            urlcolor=blue,
            }

\newcommand{\br}{\mathbf{r}}
\newcommand{\bv}{\mathbf{v}}
\newcommand{\bx}{\mathbf{x}}
\newcommand{\by}{\mathbf{y}}

\newcommand{\cD}{\mathcal{D}}
\newcommand{\cF}{\mathcal{F}}
\newcommand{\cH}{\mathcal{H}}
\newcommand{\cL}{\mathcal{L}}
\newcommand{\cV}{\mathcal{V}}
\newcommand{\cW}{\mathcal{W}}

\newcommand{\CC}{\mathbb{C}}
\newcommand{\NN}{\mathbb{N}}
\newcommand{\RR}{\mathbb{R}}
\newcommand{\ZZ}{\mathbb{Z}}

\newcommand{\eps}{\epsilon}
\newcommand{\grad}{\nabla}
\newcommand{\lam}{\lambda}
\newcommand{\lap}{\triangle}

\newcommand{\ip}[2]{\ensuremath{\left<#1,#2\right>}}

\newcommand{\image}{\operatorname{im}}
\newcommand{\onull}{\operatorname{null}}
\newcommand{\rank}{\operatorname{rank}}
\newcommand{\range}{\operatorname{range}}
\newcommand{\trace}{\operatorname{tr}}

\newcommand{\Span}{\operatorname{span}}

\newcommand{\prob}[1]{\bigskip\noindent\textbf{#1.}\quad }
\newcommand{\exer}[2]{\prob{Exercise #2 in Lecture #1}}

\newcommand{\pts}[1]{(\emph{#1 pts}) }
\newcommand{\epart}[1]{\medskip\noindent\textbf{(#1)}\quad }
\newcommand{\ppart}[1]{\,\textbf{(#1)}\quad }

\newcommand{\Matlab}{\textsc{Matlab}\xspace}
\newcommand{\Octave}{\textsc{Octave}\xspace}
\newcommand{\Python}{\textsc{Python}\xspace}
\newcommand{\Julia}{\textsc{Julia}\xspace}

\newcommand{\fl}{\operatorname{fl}}

\newcommand{\ds}{\displaystyle}

\DefineVerbatimEnvironment{mVerb}{Verbatim}{numbersep=2mm,
frame=lines,framerule=0.1mm,framesep=2mm,xleftmargin=4mm,fontsize=\footnotesize}

\newcommand{\nex}{\medskip\noindent}


\begin{document}
\scriptsize \noindent Math 617 Functional Analysis (Bueler) \hfill \emph{assigned 18 March 2024}
\normalsize\medskip

\Large\centerline{\textbf{Assignment 5}}
\large
\medskip

\centerline{\textbf{Due Wednesday 27 March 2024}}
\medskip
\normalsize

\thispagestyle{empty}

\bigskip
\noindent This Assignment is based primarily on sections 3.1, 3.2, and 3.3 of our textbook, Borthwick (2020)~\emph{Spectral Theory: Basic Concepts and Applications}, Springer.

\medskip
\noindent \textsc{Please do the following exercises.}
\smallskip


\prob{P24}  \emph{Here is why one always assumes that $\cD(T) = \cH$ for a \underline{bounded} operator.}

\medskip \noindent After the text defines ``operator'' (Definition 3.1; page 36) it says that ``a bounded operator admits a unique continuous extension to the full space $\cH$, since $\cD(T)$ is dense.''  Prove this.


\prob{P25}  \emph{This is a basic exercise for Definition 3.4, of the operator adjoint.}

\epart{a} Consider the unbounded multiplication operator $M_a$ on $\cH=\ell^2=\ell^2(\NN)$, defined for $x=(x_1,x_2,x_3,\dots)$ as
	$$M_a x = (x_1, 2 x_2, 3 x_3, \dots),$$
with a domain consisting of sequences which are eventually zero:
	$$\cD(M_a) = \left\{x \in \ell^2\,:\,\text{ there is $N$ so that if $k\ge N$ then } x_k=0\right\}.$$
Show that $\cD(M_a)$ is dense in $\cH$.

\epart{b} Directly from Definition 3.4, find a formula for the adjoint $M_a^*$, and its domain.

\epart{c} Let $M_b$ be the unbounded multiplication operator with the same formula $M_b x = (x_1, 2 x_2, 3 x_3, \dots)$ but on the larger domain
    $$\cD(M_b) = \Big\{x \in \ell^2\,:\, \sum_{k=1}^\infty k|x_k|^2 < \infty\Big\}.$$
Show that $\cD(M_b)$ is dense and that $M_b$ is self-adjoint (Definition 3.19).


\prob{P26}  \emph{Please read Example 3.6 on page 39.  In this exercise you prove a crucial aspect.}

\epart{a} Let $\cV = C^1[0,1]$ be a normed vector space with norm $\|v\| = \left(\int_0^1 |v(x)|^2\,dx\right)^{1/2}$.  Define $\omega: \to \CC$ by $\omega(v) = v(0)$.  Show that $\omega$ is linear, but also show that it is \emph{not} bounded.

\clearpage\newpage
\noindent \emph{In the next part you do \underline{not} need to prove the assertion that $C^1[0,1]$ is dense in $L^2[0,1]$, \underline{nor} do you need to prove that $\ell$ is linear.}

\epart{b} Let $\cH=L^2[0,1]$ and $\cD(T) = C^1[0,1]$ for $T = d/dx$ the first derivative.  Since $\cD(T)$ is dense in $\cH$, thus $T$ is an operator by Definition 3.1.  Fix $u\in\cD(T)$ and define the linear functional $\ell:\cD(T)\to\CC$ by
	$$\ell(v) = - \ip{Tu}{v} + \overline{u(1)} v(1) - \overline{u(0)} v(0).$$
Show that if $\ell$ is bounded in the $\cH$ norm then $u(0)=u(1)=0$.


\prob{P27}  \emph{This is a simplification of Exercise 3.12.  For both parts note that $f$, being merely measurable, could be unbounded, and indeed it could go to infinity anywhere in $(0,1)$.  However, $f(x)\in\CC$ is well-defined for every $x\in(0,1)$.  Also, note we conclude from part \emph{\textbf{(b)}} that $M_f$ is self-adjoint if and only if $f$ is real a.e.}

\medskip \noindent Let $M_f$ be a multiplication operator on $\cH=L^2(0,1)$ with $f:(0,1)\to\CC$ measurable and with domain
	$$\cD(M_f) = \left\{v \in L^2(0,1)\,:\,fv \in L^2(0,1)\right\}.$$

\epart{a}  Show that $\cD(M_f)$ is dense in $\cH$.

\epart{b}  Show that $\cD(M_f) = \cD(M_f^*)$ and that $M_f^* = M_{\overline{f}}$.

\end{document}
