\documentclass[12pt]{article}

% Layout.
\usepackage[top=1.2in, bottom=0.9in, left=1.2in, right=1.2in, headheight=1in, headsep=6pt]{geometry}

% Fonts.
\usepackage{mathptmx}
\usepackage[scaled=1.0]{helvet}
\renewcommand{\emph}[1]{\textsf{\textbf{#1}}}

% Misc packages.
\usepackage{amsmath,amssymb,latexsym,xspace}
\usepackage{graphicx,hyperref}
\usepackage{array}
\usepackage{xcolor}
\usepackage{multicol}
\usepackage{tabularx,colortbl}
\usepackage{enumitem}

\hypersetup{
    colorlinks=true,
    linkcolor=blue,
    filecolor=magenta,      
    urlcolor=blue,
    pdfauthor={Ed Bueler}
    pdftitle={Syllabus for MATH F426 Fall 2024},
    }

% Paragraph spacing
\parindent 0pt
\parskip 6pt plus 1pt
\def\tableindent{\hskip 0.5 in}
\def\ts{\hskip 1.5 em}

\usepackage{fancyhdr}
\pagestyle{fancy} 
%\chead{\large\sf\textbf{}}
\lhead{\large\sf\textbf{Syllabus MATH F426}}
\rhead{\large\sf\textbf{Fall 2024}}
  
\newcommand{\localhead}[1]{\par\smallskip\textbf{#1} \smallskip\nobreak\\}%
\def\heading#1{\localhead{\large\emph{#1}}}
\def\subheading#1{\localhead{\emph{#1}}}

\newenvironment{clist}%
{\bgroup\parskip 0pt\begin{list}{$\bullet$}{\partopsep 4pt\topsep 0pt\itemsep -2pt}}%
{\end{list}\egroup}%


\begin{document}

\strut\par%\vskip-12pt
\heading{Essential Information}

%\vskip -12pt
\strut\hbox to \hsize{\tableindent\vtop{\halign{#\hfill\ts&#\hfil\cr
\emph{Course Title}   & {\Large Numerical Analysis} \cr
\strut & \cr
\emph{Instructor}     & Ed Bueler \quad \href{mailto:elbueler@alaska.edu}{\texttt{elbueler\@@alaska.edu}} \cr
\strut & \cr
\emph{Class meeting}  & MWF 10:30--11:30 am, Chapman 106 \cr
\strut & \cr
\emph{Section CRNs}   & in-person section 901:\, 73183 \quad online section 701: 73182\cr
\strut & \cr
\emph{Public website} & \href{https://bueler.github.io/numerical/}{\texttt{bueler.github.io/numerical}}\cr
\strut & \cr
\emph{Canvas website} & \href{https://canvas.alaska.edu/courses/21626}{\texttt{canvas.alaska.edu/courses/21626}} \cr
\strut & \cr
\emph{Required text}  & A.~Greenbaum \& T.~Chartier, \textsl{Numerical Methods: Design,}\cr
 & \textsl{Analysis, and Computer Implementation of Algorithms}, \cr
 & Princeton University Press 2012\cr
}
\hfil}}

\heading{Description}
This course introduces numerical analysis using lecture, small group work in class, homework, a project, and exams.  We will learn methods for solving problems of applied mathematics on computers, and why those methods work.  We will solve problems in calculus, linear algebra, and differential equations, and in related areas.

Catalog description: \textsl{Direct and iterative solutions of systems of equations, interpolation, numerical differentiation and integration, numerical solutions of ordinary differential equations, and error analysis.}

\heading{Goals and Outcomes}
By the end of the course you will be able to evaluate and use numerical tools for solving a variety of mathematical, scientific, and engineering problems.  You will be able to program basic methods, for example for prototyping more substantial solutions.  You will have the mathematics needed for numerical approaches to new problems like optimization and partial differential equations.  Student competence with actual scientific computing, not just theory, is a goal.  You will be comfortable using Matlab or a comparable programming language.

\heading{Prerequisites}
Official prerequisites: \textsl{MATH F302 Differential Equations or MATH F314 Linear Algebra.  Recommended: Knowledge of programming.}

Regarding the two prerequisite courses, some mathematical maturity beyond calculus is more important than the specific content of either course.  However, we will indeed solve basic differential equations and linear systems by numerical means.

``Knowledge of programming'' commonly translates to some computer science course, or an engineering computer techniques course.  However, what is actually needed is willingness to learn programming in the (easy and protected) environment of Matlab.  Many students have succeeded with this as their first programming class.

\heading{Student Obligations}
Students come from math, statistics, computer science, physics, geophysics, engineering, and other majors at UAF, so perfect recall of prerequisite material is not expected.  I will devote substantial class time, especially at the beginning of the semester, to collecting the bits of needed prequisite knowledge.  I will help with getting started on Matlab programming, especially in the first week.  However, you must show initiative in a meaningful and timely way, especially when I make it clear that review is needed.

Lectures and homework together make up the core of the class.  Both the theory and practice of numerical analysis, as presented in the lecture and the textbook, will be emphasized on homework and exams.  Note that you will use the mathematical programming language Matlab, or another equivalent language, on every homework assignment, and even on the first assignment, and on your project.  Getting the most out of all parts of the class is \textsl{your} responsibility.  Please ask questions in class, both about the lecture content and about the homework assignments.

\heading{The Hybrid Classroom}
There are two sections: in-person (901; crn 73183) and online (701; crn 73182).  They are treated as one course and occur synchronously.  In this ``hybrid'' set-up, each lecture will be a recorded Zoom session generated from Chapman 106.  (The link for the Zoom session is \href{https://canvas.alaska.edu/courses/21626}{in Canvas}.  Recordings will be linked from inside Canvas only; they are not public.)  I will try to treat all students the same regarding proctored assessments and participation during class time.  Please help to make this work:
\begin{itemize}
\item \textbf{in-person students}: Please participate as energetically as you can.  I prefer for in-person students to turn in their Homework assignments on paper.
\item \textbf{online students}:  Please sign into the Zoom session, from Canvas, before class starts.  Please participate as energetically as you can, and, if possible, keep your camera on.  Regarding in-class group work, check for worksheet PDFs from the \href{https://bueler.github.io/numerical/}{public site} before class starts.  When you turn in Homework assignments electronically, please generate clear, well-ordered, and combined PDFs by email.  You will need to schedule proctoring for the Midterm and Final Exams, or attend in person on those days.
\end{itemize}

\heading{Schedule and Online Materials}
The \href{https://bueler.github.io/numerical/}{public course website} includes a \href{https://bueler.github.io/numerical/assets/general/F24/schedule.pdf}{schedule} and \href{https://bueler.github.io/numerical/daily.html}{daily log}.  The schedule includes the due date of each Homework assignment, Project due dates, and dates for the Exams.  The daily log tracks the textbook sections actually covered during each lecture, and handouts or worksheets used during class.  Please consult these frequently; they are subject to change.

Certain specific course materials will go on the \href{https://canvas.alaska.edu/courses/21626}{Canvas site}: student grades, Homework solutions, and Exam solutions.

\heading{Office Hours and Communication}
My office hours are shown online at \href{http://bueler.github.io/OffHrs.htm}{\texttt{bueler.github.io/OffHrs.htm}}; I hold them in Chapman 306C.  Students can also schedule meetings with me outside of these hours.

I will use Canvas to send announcements.  If I need to contact you outside of class times, I will email via Canvas.  Please set your email address in Canvas appropriately.

\heading{Evaluation and Grades}
\vskip -10pt

\begin{tabular}{|l|l|r|}
\hline
Homework         & nearly weekly & 40\% \\
\hline
Project Proposal & start of class on 1 November & 5\% \\
\hline
Project          & start of class on 6 December & 10\% \\
\hline
Midterm Exam     & in-class; 11 October & 20\%  \\
\hline
Final Exam       & in-class; 10:15am-12:15pm Wednesday 11 December & 25\% \\
\hline
total            & & 100\% \\
\hline
\end{tabular}

Scores may be adjusted based on the actual difficulty of the work and/or on average class performance, and adjustments will be applied to all students equally.  The scores of the various parts will be summed and the final course grade will be assigned as follows.

\begin{tabular}{llllll}
A  & 93--100\% & B- & 79--81\%  & D+ & 65--67\%  \\
A- & 90--92\%  & C+ & 76--78\%  & D  & 60--64\%  \\
B+ & 87--89\%  & C  & 68--75\%  & D- & 57--59\%  \\
B  & 82--86\%  & C- & not given & F  & $\le$ 56\%
\end{tabular}

These ranges are a guarantee and a lower bound.  I reserve the right to increase your grade above these ranges based on the actual difficulty of the work and/or on average class performance.  Any such increases will preserve grade ordering by weighted total score.

\heading{Homework}
Homework is due at the start of class on the date announced on the \href{https://bueler.github.io/numerical/assets/general/F24/schedule.pdf}{schedule}.  \emph{Late homework is not accepted.}  If you have unavoidable circumstances which do not allow you to turn in an Assignment on time then please contact me (\href{mailto:elbueler@alaska.edu}{\texttt{elbueler\@@alaska.edu}}) in advance. 

\heading{Project}
You will get to choose the topic of your Project.  I will announce the Project rules in mid-October.  You will turn in a Project Proposal on Friday 1 November, and I will give feedback on that.  The final form of your Project is due in-class on the last regular lecture day, Friday 6 December.

\heading{Exams}
There will be two in-class exams, a Midterm and a Final.  See the \href{https://bueler.github.io/numerical/assets/general/F24/schedule.pdf}{schedule} for dates.  Each will cover problems from identified sections which you have already seen on Homework.

A make-up Midterm Exam will be given only for documented circumstances, at my discretion.  Department policy (below) does not allow me to move the time of the Final Exam.

%\clearpage\newpage
%\phantom{foo}
\heading{Internet and AI usage, and other assistance}
You will not have access to AI tools, or the internet, or any other tools except a writing implement, during the Midterm and Final Exams.  These assessments are proctored and on-paper, and represent 45\% of your course grade.

Regarding Homework and your Project, you are encouraged to talk to other students about the problems, and to use available tools appropriately.  However, the work you turn in must be your own.  Please do not copy proofs or solutions from online sources, from searching or generated via AI tools like ChatGPT.  If I detect this then I have the right to give you a zero on that assignment, or to require that you explain the solution during my office hours.

When you do get assistance, it goes without saying that \textsl{doing your own thinking will have the greatest benefits}.  Fully understand the material you turn in, even if hints from other sources were used in generating them.

\small
\heading{Rules and Policies}
\vskip -20pt

\subheading{Incomplete Grade} 
Incomplete (I) will only be given in
  DMS courses in cases where
  the student has completed the majority (normally all but the last
  three weeks) of a course with a grade of C or better, but for
  personal reasons beyond his/her control has been unable to complete
  the course during the regular term. Negligence or indifference are
  not acceptable reasons for granting an incomplete grade.

\subheading{Late Withdrawals} 
A withdrawal after the deadline from a DMS course will
  normally be granted only in cases where the student is performing
  satisfactorily (i.e., C or better) in a course, but has exceptional
  reasons, beyond his/her control, for being unable to complete the
  course.  These exceptional reasons should be detailed in writing to
  the instructor, Department Chair and the Dean.

\subheading{No Early Final Examinations}
Final examinations for DMS courses shall not be held earlier than the date and time published in the official term schedule.  Normally, a student will not be allowed to take a final exam early.  Exceptions can be made by individual instructors, but should only be allowed in exceptional circumstances and in a manner which doesn't endanger the security of the exam.

\subheading{Academic Dishonesty}
Academic dishonesty, including cheating and plagiarism, will not be tolerated.  It is a violation of the Student Code of Conduct and will be punished according to UAF procedures.

\subheading{Student protections and service statement}
Every qualified student is welcome in my classroom.  As needed, I am happy to work with you, Disability Services, Veterans' Services, Rural Student Services, and so on, to find reasonable accommodations.  Students at this University are protected against sexual harassment and discrimination (Title IX), and minors have additional protections.  For more information on your rights as a student and the resources available to you to resolve problems, please go the following site: \href{https://www.uaf.edu/handbook/}{\texttt{www.uaf.edu/handbook}}.

\hfill  \scriptsize [syllabus version: \today] \normalsize

\end{document}
