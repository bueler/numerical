\documentclass[11pt]{amsart}
%prepared in AMSLaTeX, under LaTeX2e
\addtolength{\oddsidemargin}{-.6in} 
\addtolength{\evensidemargin}{-.6in}
\addtolength{\topmargin}{-.6in}
\addtolength{\textwidth}{1.4in}
\addtolength{\textheight}{1.2in}

\renewcommand{\baselinestretch}{1.1}

\usepackage{verbatim,fancyvrb}
\usepackage{xspace}
\usepackage{palatino}

\usepackage{listings}             % Include the listings-package
\lstset{language=Matlab}          % Set your language

\usepackage[final]{graphicx}

\usepackage{amssymb}

\usepackage[pdftex, colorlinks=true, plainpages=false, linkcolor=black, citecolor=red, urlcolor=red]{hyperref}

\newtheorem*{thm}{Theorem}
\newtheorem*{defn}{Definition}
\newtheorem*{example}{Example}
\newtheorem*{problem}{Problem}
\newtheorem*{remark}{Remark}

% macros
\newcommand{\br}{\mathbf{r}}
\newcommand{\bv}{\mathbf{v}}
\newcommand{\bx}{\mathbf{x}}
\newcommand{\by}{\mathbf{y}}

\newcommand{\CC}{\mathbb{C}}
\newcommand{\RR}{\mathbb{R}}
\newcommand{\ZZ}{\mathbb{Z}}

\newcommand{\eps}{\epsilon}
\newcommand{\grad}{\nabla}
\newcommand{\lam}{\lambda}
\newcommand{\lap}{\triangle}

\newcommand{\ip}[2]{\ensuremath{\left<#1,#2\right>}}

\newcommand{\onull}{\operatorname{null}}
\newcommand{\rank}{\operatorname{rank}}
\newcommand{\range}{\operatorname{range}}
\newcommand{\cond}{\operatorname{cond}}

\newcommand{\Matlab}{\textsc{Matlab}\xspace}
\newcommand{\Octave}{\textsc{Octave}\xspace}
\newcommand{\Python}{\textsc{Python}\xspace}

\newcommand{\mfile}[2]{
	\bigskip
	\begin{quote}
		\bigskip
		\VerbatimInput[frame=single,framesep=3mm,label=\fbox{\normalsize \textsl{\,#1\,}},fontfamily=courier,fontsize=\scriptsize]{#2}
		\bigskip
	\end{quote}
}

\DefineVerbatimEnvironment{mVerb}{Verbatim}{numbersep=2mm,frame=lines,framerule=0.1mm,framesep=2mm,xleftmargin=4mm,fontsize=\small}

\newcommand{\prob}[1]{\bigskip\noindent\textbf{P#1.}\quad }
\newcommand{\pts}[1]{(\emph{#1 pts}) }
\newcommand{\epart}[1]{\medskip\noindent\textbf{#1)}\quad }
\newcommand{\ppart}[1]{\,\textbf{#1)}\quad }

\newcommand*\circled[1]{\tikz[baseline=(char.base)]{
            \node[shape=ellipse,draw,inner sep=2pt] (char) {#1};}}
\newcommand{\tb}{R.~LeVeque, \emph{Finite Difference Methods for Ordinary and Partial Differential Equations}}


\begin{document}
\scriptsize \noindent Math 426 Numerical Analysis (Bueler) \hfill 25 October 2024
\bigskip

\Large\centerline{\textbf{your Math 426 project}}
\normalsize

\thispagestyle{empty}

\bigskip

\subsection*{Overview}  The goal of your Math 426 project is to get practical experience in numerical analysis by doing more substantial work on a topic that you would in a homework problem.  The topic, which should be different from those we have already covered, or will cover, is chosen by you.  I am happy to consult on the topic, or any other aspect of your Project.  Feel free to ask other advisors or professors, etc., for topic suggestions.  The Project is not intended to be a terribly big thing (see below for length expectations), but rather an opportunity to look in a little detail at a new-to-you kind of problem and some numerical algorithm(s) which solve it.  It should be at least a little bit of fun!  You are expected to write about the subject---essentially an essay, but with equations and images where appropriate---and run a small example.

\section*{Topic suggestions}

This suggestion list is \emph{not} exhaustive; you can find other topics.  These appear alphabetically, not in order of importance or difficulty etc.
\begin{itemize}
\item banded matrix systems (tridiagonal, etc.): algorithms for solving
\item \texttt{chebfun.org}
\item computer graphics calculation(s)
\item eigenvalues: algorithms for finding
\item fast Fourier transform
\item finite element (triangular meshes, and the method)
\item global position system (GPS) computations
\item integration in multiple variables
\item iterated function system fractals
\item linear programming
\item machine learning: neural networks
\item machine learning: support vector machines or decision trees
\item Markov chains
\item Monte Carlo methods for integration
\item Newton's method fractals
\item optimization algorithms: steepest descent or Newton's method
\item overdetermined systems: least squares and regression
\item partial differential equations: Laplace, heat, wave, or Schr\"odinger equation
\item QR factorization of matrices
\item random number generators
\item splines (in addition to cubic; e.g.~B\'ezier curves, B-splines, NURBS)
\item stochastic processes
\item time series: filters
\end{itemize}

\medskip \noindent Please do a web search for topics you find interesting, and at least look for a wikipedia page.  There are some excellent YouTube videos out there.  Also, Chapter 1 of our textbook\footnote{Greenbaum \& Chartier, \emph{Numerical Methods: Design, \dots}, Princeton University Press 2012.} is a reasonable source for ideas.  The intrinsic difficulty of the standard examples of these problems varies quite a bit. Nonetheless, I think I am able to advise on what you can do in the time available.


\section*{Content and length of Project Proposal}   

Please write a \textbf{one page} (at most) Project Proposal, and turn it in on \textbf{Monday 4 November at the start of class}.

This Proposal should contain:
\begin{enumerate}
\item Describe in several paragraphs what is your topic.  Address the main mathematical ideas, and identify one or two applications.  Why can't the problem be solved by hand?  What kinds of guarantees/arguments/proofs tell you about accuracy or speed?
\item Identify/state a basic example which you will actually compute.  For this example, you will write and run some Matlab/etc.~code in the final version of your Project.  (However, code is not expected in the Proposal.)  Try to identify one of the simplest examples which truly reflects the topic you have chosen.  You will have to use your judgment about which ``black boxes'' make sense in context, but generally try to avoid them.  (For example, I don't necessarily expect you to write your own implementation of the Fast Fourier Transform (FFT) yourself, but if not you should plan to apply Matlab's \texttt{fft} to a couple of different kinds of problems related to extracting frequencies etc.  Or you could implement the FFT, and check for correctness, but not worry much about applying it.)
\item List 2 or 3 sources for your proposal.  Given URLs for websites if appropriate, or journal article or book citations, or (if appropriate) where it is covered in the book.
\end{enumerate}

Originality is \textbf{not} required for your Project.  So feel free to propose to write and test a code even in a case where you already have good available models to follow.

You should probably write your Proposal in a digital form, which can then be a starting draft which you modify into the final Project.  However, please turn your Proposal in on paper, and likewise the final Project.  (Except if you are attending online: emailed PDFs are fine.)

After you turn in the Proposal, I will give you feedback which is intended to improve your grade on the complete Project.  I am happy to talk to you about your Project at any stage.

The Proposal is worth 5\% of your course grade.  Please take it seriously, recognizing that a good Proposal will lead to an easier-to-complete Project.


\section*{Content and length of of the final Project}  

Please write up your Project as a document of \textbf{between 10 and 20 pages}.  Turn it, on paper (unless attending online) on \textbf{Friday 6 December, either at the start of class or by 5pm in my office box}.  Please make your paper document double-sided if possible.  Please make it organized and professional looking.  Include your name, the date, and the title near the top of the first page.

Use reasonable section headings; recommended headings are \emph{Introduction}, \emph{Example Problem(s)}, \emph{Algorithm(s)}, \emph{Analysis}, \emph{Results}, and \emph{References}.  The Algorithm(s) section should show the code you wrote.  Note that ``Analysis'' should address either how accurate the method is, or how fast it runs, or both; in some cases analysis in advance is difficult, but you can analyze the results of your numerical experiments, for correctness, accuracy, or speed.

In your Project, fill out the skeleton you have already proposed, taking into account my feedback on your Proposal.  You may need to alter your earlier plans!  It is fine, and often appropriate, for the final Project to have different computational or analytical details, or a different kind of example, than proposed.

I expect your work to be your own.  If it looks like you are mindlessly copying things which you don't understand then I will go ahead and ask you to explain what you did in person.

The completed Project is worth 10\% of your course grade.

\end{document}
