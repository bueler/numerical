\documentclass[12pt]{amsart}
%prepared in AMSLaTeX, under LaTeX2e
\addtolength{\oddsidemargin}{-.6in} 
\addtolength{\evensidemargin}{-.6in}
\addtolength{\topmargin}{-.7in}
\addtolength{\textwidth}{1.3in}
\addtolength{\textheight}{1.3in}

\renewcommand{\baselinestretch}{1.05}

\usepackage{verbatim} % for "comment" environment

\usepackage{palatino}

\usepackage[final]{graphicx}

\usepackage{tikz}
\usetikzlibrary{positioning}

\usepackage{enumitem}

\newtheorem*{thm}{Theorem}
\newtheorem*{defn}{Definition}
\newtheorem*{example}{Example}
\newtheorem*{problem}{Problem}
\newtheorem*{remark}{Remark}

\newcommand{\mtt}{\texttt}
\usepackage{alltt,xspace}
\newcommand{\mfile}[1]
{\medskip\begin{quote}\scriptsize \begin{alltt}\input{#1.m}\end{alltt} \normalsize\end{quote}\medskip}

% macros
\usepackage{amssymb}
\newcommand{\bA}{\mathbf{A}}
\newcommand{\bB}{\mathbf{B}}
\newcommand{\bE}{\mathbf{E}}
\newcommand{\bF}{\mathbf{F}}
\newcommand{\bJ}{\mathbf{J}}
\newcommand{\br}{\mathbf{r}}
\newcommand{\bx}{\mathbf{x}}
\newcommand{\hbi}{\mathbf{\hat i}}
\newcommand{\hbj}{\mathbf{\hat j}}
\newcommand{\hbk}{\mathbf{\hat k}}
\newcommand{\hbn}{\mathbf{\hat n}}
\newcommand{\hbr}{\mathbf{\hat r}}
\newcommand{\hbt}{\mathbf{\hat t}}
\newcommand{\hbx}{\mathbf{\hat x}}
\newcommand{\hby}{\mathbf{\hat y}}
\newcommand{\hbz}{\mathbf{\hat z}}
\newcommand{\hbphi}{\mathbf{\hat \phi}}
\newcommand{\hbtheta}{\mathbf{\hat \theta}}
\newcommand{\complex}{\mathbb{C}}
\newcommand{\ppr}[1]{\frac{\partial #1}{\partial r}}
\newcommand{\ppt}[1]{\frac{\partial #1}{\partial t}}
\newcommand{\ppx}[1]{\frac{\partial #1}{\partial x}}
\newcommand{\ppy}[1]{\frac{\partial #1}{\partial y}}
\newcommand{\ppz}[1]{\frac{\partial #1}{\partial z}}
\newcommand{\pptheta}[1]{\frac{\partial #1}{\partial \theta}}
\newcommand{\ppphi}[1]{\frac{\partial #1}{\partial \phi}}
\newcommand{\Div}{\ensuremath{\nabla\cdot}}
\newcommand{\Curl}{\ensuremath{\nabla\times}}
\newcommand{\curl}[3]{\ensuremath{\begin{vmatrix} \hbi & \hbj & \hbk \\ \partial_x & \partial_y & \partial_z \\ #1 & #2 & #3 \end{vmatrix}}}
\newcommand{\cross}[6]{\ensuremath{\begin{vmatrix} \hbi & \hbj & \hbk \\ #1 & #2 & #3 \\ #4 & #5 & #6 \end{vmatrix}}}
\newcommand{\eps}{\epsilon}
\newcommand{\grad}{\nabla}
\newcommand{\image}{\operatorname{im}}
\newcommand{\integers}{\mathbb{Z}}
\newcommand{\ip}[2]{\ensuremath{\left<#1,#2\right>}}
\newcommand{\lam}{\lambda}
\newcommand{\lap}{\triangle}
\newcommand{\note}[1]{[\scriptsize #1 \normalsize]}
\newcommand{\MatIN}[1]{\mtt{>> #1}}
\newcommand{\onull}{\operatorname{null}}
\newcommand{\rank}{\operatorname{rank}}
\newcommand{\range}{\operatorname{range}}
\renewcommand{\P}{\mathcal{P}}
\newcommand{\real}{\mathbb{R}}
\newcommand{\trace}{\operatorname{tr}}

\newcommand{\prob}[1]{\bigskip\noindent\textbf{#1.}\quad }
\newcommand{\exer}[2]{\prob{Exercise #2 on page #1}}
\newcommand{\exerpages}[2]{\prob{Exercise #2 on pages #1}}

\newcommand{\pts}[1]{(\emph{#1 pts}) }
\newcommand{\epart}[1]{\bigskip\noindent\textbf{(#1)}\quad }
\newcommand{\ppart}[1]{\,\textbf{(#1)}\quad }

\newcommand{\Matlab}{\textsc{Matlab}\xspace}
\newcommand{\Octave}{\textsc{Octave}\xspace}
\newcommand{\MO}{\textsc{Matlab}/\textsc{Octave}\xspace}

\newcommand{\bitboxes}{
\large
\begin{center}
\begin{tabular}{|c|c|c|c|c|c|c|c|c|c|c|c|} \hline
\phantom{$e_1$} &
\phantom{$e_1$} &
\phantom{$e_1$} &
\phantom{$e_1$} &
\phantom{$e_1$} &
\phantom{$e_1$} &
\phantom{$e_1$} &
\phantom{$e_1$} &
\phantom{$e_1$} &
\phantom{$e_1$} &
\phantom{$e_1$} &
\phantom{$e_1$} \\ \hline
\end{tabular}
\end{center}
\normalsize}


\begin{document}
\scriptsize \noindent Math 426 Numerical Analysis (Bueler) \hfill \fbox{\emph{Not to be turned in!}}
\normalsize\medskip

\Large\centerline{\textbf{Worksheet: If IEEE 754 had a 12-bit standard \dots}}
\medskip
\normalsize

\thispagestyle{empty}

The actual IEEE 754 standards for 64-bit double precision representation is cumbersome to play with, so for convenience we pretend here that the standard has a 12-bit version.  It might look like this:

\medskip\large
\begin{center}
\begin{tabular}{|c|c|c|c|c|c|c|c|c|c|c|c|} \hline
$s$ & $e_1$ & $e_2$ & $e_3$ & $e_4$ & $b_1$ & $b_2$ & $b_3$ & $b_4$ & $b_5$ & $b_6$ & $b_7$ \\ \hline
\end{tabular}
\end{center}
\medskip\normalsize

\noindent These 12 bits are organized to represent a \emph{nonzero} number:
\medskip\large
	$$x = (-1)^s\,(1.b_1 b_2 b_3 b_4 b_5 b_6 b_7)_2 \,\, 2^{(e_1 e_2 e_3 e_4)_2 - (0111)_2}$$
\normalsize

\noindent Note that $(1.b_1 b_2 b_3 b_4 b_5 b_6 b_7)_2$ is called the \emph{mantissa}.  The power on the big $2$ is the \emph{exponent}.  The special offset $(0111)_2$, equal to 7 in base ten, is called the \emph{exponent bias}.  We also define some exceptional cases:
\begin{itemize}
\item exponent bits $(0000)_2$ define the number zero or subnormal numbers
\item exponent bits $(1111)_2$ define the other exceptions: $\pm\infty$ and NaN
\end{itemize}
(No further details of the $(1111)_2$ exceptions will be considered here.)

\epart{a}  What is the largest real number that this system can represent?  Show the bits.

\bitboxes
\vfill

\epart{b}  What is the smallest positive number that this system can represent?  (\emph{What is the first normal number to the right of zero?})  Show the bits.

\bitboxes
\vfill

\epart{c}  If we define $\eps_{\text{machine}}$ as the gap between $1$ and the next representable number greater than 1, what is the value of $\eps_{\text{machine}}$ in this system?
\vfill

\epart{d}  What is the representation of zero?  (\emph{Of $+0$? Of $-0$?})  Show the bits.

\bitboxes
\vfill

\newpage
\epart{e}  What is the representation of 4?  Show the bits.

\bitboxes
\vfill

\epart{f}  What is the largest representable number which is smaller than 8?  Show the bits.

\bitboxes
\vfill

\epart{g}  In the interval $[4,8)$, how many numbers can be represented?
\vfill

\epart{h}  What is the closest number to $\pi=(3.14159\cdots)_{10}$?  Show the bits.

\bitboxes
\vfill

\epart{i}  Exactly how many distinct non-exceptional numbers can be represented in this system?  (\emph{Include the number zero but exclude subnormal numbers and any exceptions using exponent $(1111)_2$, i.e.~$\pm\infty$ and NaN.})
\vfill

\epart{j}  Show the bits of one subnormal number.

\bitboxes
\vspace{0.6in}

\end{document}
