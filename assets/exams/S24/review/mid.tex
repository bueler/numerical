\documentclass[12pt]{amsart}
\addtolength{\topmargin}{-0.4in} % usually -0.25in
\addtolength{\textheight}{0.9in} % usually 1.25in
\addtolength{\oddsidemargin}{-0.8in}
\addtolength{\evensidemargin}{-0.8in}
\addtolength{\textwidth}{1.5in} %\setlength{\parindent}{0pt}

\newcommand{\normalspacing}{\renewcommand{\baselinestretch}{1.1}\tiny\normalsize}
\newcommand{\bigspacing}{\renewcommand{\baselinestretch}{1.21}\tiny\normalsize}
\newcommand{\tablespacing}{\renewcommand{\baselinestretch}{1.0}\tiny\normalsize}
\normalspacing

% macros
\usepackage{amssymb,xspace}
\usepackage[pdftex,colorlinks=true]{hyperref}

\usepackage[final]{graphicx}
\newcommand{\regfigure}[3]{\includegraphics[height=#2in,width=#3in]{#1.eps}}

\newtheorem*{thm}{Theorem}
\newtheorem*{lem}{Lemma}
\usepackage{fancyvrb}

\newcommand{\bu}{\mathbf{u}}
\newcommand{\bv}{\mathbf{v}}

\newcommand{\cH}{\mathcal{H}}
\newcommand{\cL}{\mathcal{L}}

\newcommand{\CC}{{\mathbb{C}}}
\newcommand{\RR}{{\mathbb{R}}}
\newcommand{\ZZ}{{\mathbb{Z}}}
\newcommand{\ZZn}{{\mathbb{Z}}_n}
\newcommand{\NN}{{\mathbb{N}}}

\newcommand{\eps}{\epsilon}
\newcommand{\lam}{\lambda}
\newcommand{\ip}[2]{\left<#1,#2\right>}
\newcommand{\erf}{\operatorname{erf}}

\renewcommand{\Im}{\operatorname{Im}}
\renewcommand{\Re}{\operatorname{Re}}

\newcommand{\Span}{\operatorname{span}}
\newcommand{\rank}{\operatorname{rank}}
\newcommand{\range}{\operatorname{range}}
\newcommand{\trace}{\operatorname{tr}}
\newcommand{\Null}{\operatorname{null}}

\newcommand{\ds}{\displaystyle}

\newcommand{\Matlab}{\textsc{Matlab}\xspace}

\newcommand{\prob}[1]{\bigskip\noindent\large\textbf{#1.} \normalsize}
\newcommand{\bookprob}[1]{\bigskip\noindent\large\textbf{Exercise #1.} \normalsize}
\newcommand{\probpart}[1]{\smallskip\noindent\textbf{(#1)}\quad }
\newcommand{\aprobpart}[1]{\textbf{(#1)}\quad }

\newcommand{\textbook}{D.~Borthwick, \emph{Spectral Theory: Basic Concepts and Applications}, GTM 284, Springer 2020}


\begin{document}
\scriptsize \noindent Math 617 Functional Analysis (Bueler) \hfill \today
\thispagestyle{empty}

\bigskip
\LARGE\centerline{\textbf{Review Guide}}

\medskip
\Large\centerline{for in-class \textbf{Midterm Quiz} on \textbf{Wednesday March 20}}

\normalsize
\bigskip
\begin{quote}
The Midterm Quiz will be built from those parts of Chapter 2 in Borthwick\footnote{\textbook.} which we have actually covered, and from the Handout.\footnote{\href{https://bueler.github.io/fa/assets/handouts/getstarted.pdf}{\,\texttt{bueler.github.io/fa/assets/handouts/getstarted.pdf}}}  In Borthwick we covered 2.1, 2.2, 2.3, 2.4, 2.6, and 2.7 pretty thoroughly, except for 2.3.2.  You will not be asked about 2.5 (Sobolev spaces) or 2.3.2.

The problems will be of these types, based on the lists below: state definitions, state theorems, describe or illustrate geometrical ideas, apply theorems in easy situations, and prove simple theorems/corollaries.
\end{quote}
\bigskip

\bigspacing
\noindent \textbf{Definitions}.  Be able to state the precise definition.
\begin{itemize}
\item metric $d(\cdot,\cdot)$
\item norm $\|\cdot\|$
\item open and closed subsets of a metric space
\item compact set
\item limit of a sequence (in a metric space)
\item continuous $\CC$-valued function (on a metric space)
\item Cauchy sequence (in a metric space)
\item sequential compactness of a set (in a metric space; p.~30)
\item complete metric space
\item Banach space
\item linear map between vector spaces
\item eigenvector and eigenvalue for a linear map on a vector space
\item inner product $\ip{\cdot}{\cdot}$ on a complex vector space
\item norm in an inner product space
\item complex Hilbert space $\cH$
\item $\ell^p=\ell^p(\NN)$ spaces for $1\le p \le \infty$, including norm
\item positive measure
\item almost everywhere
\item simple function
\item $L^p(X)=L^p(X,dm)$\footnote{For $1\le p \le \infty$ and $dm=(\text{Lebesgue measure})$, $L^p(X)=L^p(X,dm)$ can be defined for any Lebesgue-measurable set $X\subset \RR^n$.  Note that $L^p(X,dm)$ is a set of \emph{measurable} functions $f:X\to\CC$.} for $1\le p < \infty$, including norm
\item $L^\infty(X)=L^\infty(X,dm)$, including norm
\item $C^\infty(\RR)$, $C^\infty(a,b)$, $C_0^\infty(\RR)$, $C_0^\infty(a,b)$
\item bounded operator between normed vector spaces
\item $\cL(V,W)$ for $V,W$ normed vector spaces
\item dual space $W'$, for $W$ a normed vector space
\item operator norm
\item multiplication operator $M_f$ on $L^p(X,d\mu)$, for $f\in L^\infty(X,d\mu)$
\item kernel and range of $T \in \cL(V,W)$
\item linear isometry between normed vector spaces (p.~11)
\item left and right shift operators on $\ell^2$
\item $S^\perp$, for $S\subset \cH$ any subset
\item direct sum\footnote{We write $V\oplus W=X$ if: \quad \emph{i}) $X$ is a vector space, \emph{ii}) $V,W$ are subspaces of $X$, \emph{iii}) for each $x\in X$ there exists $v\in V$ and $w\in W$ so that $v+w=x$ (i.e.~$V+W=X$), and \emph{iv}) $V\cap W = \{0\}$.} $V\oplus W=X$ of subspaces
\item separable Hilbert space
\item $w_n \to w$ weakly in $\cH$ (p.~27)
\item orthonormal (ON) sequence in $\cH$
\item ON basis in $\cH$
\end{itemize}

\normalspacing
\medskip
\noindent \textbf{Theorems and Lemmas}.  Know the precise statement of the theorem.  Be able to prove if so indicated.
\begin{itemize}
\item Heine-Borel theorem for $\RR^n$ (Handout)
\item extreme value theorem for $\RR$-valued functions on compact $K\subset \RR^n$ (Handout)
\item Fundamental Theorem of Calculus (Handout)
\item $\cL(V,W)$ is complete in operator norm if $W$ is complete (Theorem 2.10)
\item Cauchy-Schwarz (Theorem 2.15)
\item parallelogram law (Lemma 2.17; \textbf{be able to prove})
\item if $W\subset \cH$ is closed then $\cH = W\oplus W^\perp$ (Theorem 2.27)
\item Riesz lemma (Theorem 2.28)
\item $\cH$ separable $\iff$ there exists an ON basis of $\cH$ (Theorem 2.33)
\item Bessel's inequality (Theorem 2.34)
\end{itemize}

\normalspacing
\medskip
\noindent \textbf{Techniques}.  Be able to justify and use these calculation/proof techniques.
\begin{itemize}
\item To show that $W\subset X$ is a subspace of a complex vector space it suffices to show that $0\in W$ and that for $v,w\in W$ and $\lambda \in \CC$ then $v + \lambda w \in W$.
\item If $\cH$ is a Hilbert space and $\{e_j\}_{j\in\NN}$ is an ON basis of $\cH$ then for $v\in\cH$  there are coefficients $c_j\in\CC$ so that $\ds v = \sum_{j=1}^\infty c_j e_j$, and furthermore $c_j=\ip{e_j}{v}$.
\item The triangle inequality $\|x+y\|\le \|x\|+\|y\|$ holds in a normed vector space, and its corollary $\big|\|x\|-\|y\|\big| \le \|x-y\|$ as well.
\item The norm in a normed vector space is continuous.
\item The inner product in an inner product space is continuous.
\end{itemize}

\end{document}

