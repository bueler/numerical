\documentclass[12pt]{amsart}
\addtolength{\topmargin}{-0.6in} % usually -0.25in
\addtolength{\textheight}{1.2in} % usually 1.25in
\addtolength{\oddsidemargin}{-0.7in}
\addtolength{\evensidemargin}{-0.7in}
\addtolength{\textwidth}{1.5in} %\setlength{\parindent}{0pt}

\newcommand{\normalspacing}{\renewcommand{\baselinestretch}{1.1}\tiny\normalsize}
\newcommand{\bigspacing}{\renewcommand{\baselinestretch}{1.21}\tiny\normalsize}
\newcommand{\tablespacing}{\renewcommand{\baselinestretch}{1.0}\tiny\normalsize}
\normalspacing

% macros
\usepackage{amssymb,xspace}
\usepackage[pdftex,colorlinks=true]{hyperref}

\usepackage[final]{graphicx}
\newcommand{\regfigure}[3]{\includegraphics[height=#2in,width=#3in]{#1.eps}}

\newtheorem*{thm}{Theorem}
\newtheorem*{lem}{Lemma}

\newcommand{\mtt}{\texttt}
\newcommand{\mtl}[1]{{\texttt{>>#1}}}
\usepackage{alltt}
\usepackage{fancyvrb}

\newcommand{\CC}{{\mathbb{C}}}
\newcommand{\RR}{{\mathbb{R}}}
\newcommand{\ZZ}{{\mathbb{Z}}}
\newcommand{\ZZn}{{\mathbb{Z}}_n}
\newcommand{\NN}{{\mathbb{N}}}
\newcommand{\eps}{\epsilon}
\newcommand{\lam}{\lambda}
\newcommand{\bu}{\mathbf{u}}
\newcommand{\bv}{\mathbf{v}}
\newcommand{\ip}[2]{\mathrm{\left<#1,#2\right>}}
\newcommand{\erf}{\operatorname{erf}}

\renewcommand{\Re}{\operatorname{Re}}

\newcommand{\Span}{\operatorname{span}}
\newcommand{\rank}{\operatorname{rank}}
\newcommand{\range}{\operatorname{range}}
\newcommand{\trace}{\operatorname{tr}}
\newcommand{\Null}{\operatorname{null}}

\newcommand{\Matlab}{\textsc{Matlab}\xspace}
\newcommand{\Octave}{\textsc{Octave}\xspace}
\newcommand{\pylab}{\textsc{pylab}\xspace}
\newcommand{\longMOP}{\textsc{Matlab}\big|\textsc{Octave}\big|\textsc{pylab}\xspace}
\newcommand{\MOP}{\textsc{M}\big|\textsc{O}\big|\textsc{p}\xspace}

\newcommand{\prob}[1]{\bigskip\noindent\large\textbf{#1.} \normalsize}
\newcommand{\bookprob}[1]{\bigskip\noindent\large\textbf{Exercise #1.} \normalsize}
\newcommand{\probpart}[1]{\smallskip\noindent\textbf{(#1)}\quad }
\newcommand{\aprobpart}[1]{\textbf{(#1)}\quad }

\newcommand{\textbook}{D.~Borthwick, \emph{Spectral Theory: Basic Concepts and Applications}, GTM 284, Springer 2020}

\begin{document}
\scriptsize \noindent Math 617 Functional Analysis (Bueler) \hfill \today
\thispagestyle{empty}

\bigskip
\LARGE\centerline{\textbf{Review Guide}}

\medskip
\Large\centerline{for in-class \textbf{Midterm Quiz} on \textbf{Wednesday March 22}}

\normalsize
\bigskip
\begin{quote}
The Midterm Quiz will be built from those parts of Chapter 2 in Borthwick\footnote{\textbook.} which we have actually covered, and from the Handout on definitions and facts.\footnote{\href{https://bueler.github.io/fa/assets/handouts/getstarted.pdf}{\,\texttt{bueler.github.io/fa/assets/handouts/getstarted.pdf}}}  In Borthwick we covered 2.1, 2.2, 2.3, 2.4, 2.6, and 2.7 thoroughly, except for 2.3.2.  You will not be asked about 2.5 (Sobolev spaces).

The problems will be of these types: state definitions, state theorems, describe or illustrate geometrical ideas, basic applications of theorems, quick calculations, prove simple theorems/corollaries.
\end{quote}
\bigskip

\bigspacing
\noindent \textbf{Definitions}. Know how to define:
\begin{itemize}
\item X
\end{itemize}

\normalspacing

\noindent \textbf{Theorems}.  Know as facts.  Be able to prove unless otherwise stated.
\begin{itemize}
\item X
\end{itemize}

\normalspacing

\noindent \textbf{Facts and Formulas}.  Know as facts.  Be able to prove unless otherwise stated.
\begin{itemize}
\item X
\end{itemize}

\normalspacing

\bigskip\noindent \textbf{Ideas}.  Please be comfortable with the following ideas!  Some ideas correspond to theorems, but otherwise it is just a perspective or paradigm.

\bigspacing
\begin{itemize}
\item X
\end{itemize}


\end{document}

