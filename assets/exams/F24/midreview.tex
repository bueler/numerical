\documentclass[12pt]{amsart}
\addtolength{\topmargin}{-0.6in} % usually -0.25in
\addtolength{\textheight}{1.1in} % usually 1.25in
\addtolength{\oddsidemargin}{-0.7in}
\addtolength{\evensidemargin}{-0.7in}
\addtolength{\textwidth}{1.5in} %\setlength{\parindent}{0pt}

\newcommand{\normalspacing}{\renewcommand{\baselinestretch}{1.05}\tiny\normalsize}
\newcommand{\bigspacing}{\renewcommand{\baselinestretch}{1.13}\tiny\normalsize}
\newcommand{\tablespacing}{\renewcommand{\baselinestretch}{1.0}\tiny\normalsize}
\normalspacing

% macros
\usepackage{amssymb,xspace}
\usepackage[pdftex,colorlinks=true]{hyperref}

\usepackage[final]{graphicx}
\newcommand{\regfigure}[3]{\includegraphics[height=#2in,width=#3in]{#1.eps}}

\newtheorem*{thm}{Theorem}
\newtheorem*{lem}{Lemma}

\newcommand{\mtt}{\texttt}
\newcommand{\mtl}[1]{{\texttt{>>#1}}}
\usepackage{alltt}
\usepackage{fancyvrb}

\newcommand{\bb}{\mathbf{b}}
\newcommand{\bu}{\mathbf{u}}
\newcommand{\bv}{\mathbf{v}}
\newcommand{\bx}{\mathbf{x}}

\newcommand{\CC}{{\mathbb{C}}}
\newcommand{\RR}{{\mathbb{R}}}
\newcommand{\ZZ}{{\mathbb{Z}}}
\newcommand{\ZZn}{{\mathbb{Z}}_n}
\newcommand{\NN}{{\mathbb{N}}}

\newcommand{\eps}{\epsilon}
\newcommand{\grad}{\nabla}
\newcommand{\lam}{\lambda}
\newcommand{\ip}[2]{\mathrm{\left<#1,#2\right>}}
\newcommand{\erf}{\operatorname{erf}}

\renewcommand{\Re}{\operatorname{Re}}
\renewcommand{\Im}{\operatorname{Im}}
\newcommand{\Arg}{\operatorname{Arg}}

\newcommand{\Span}{\operatorname{span}}
\newcommand{\rank}{\operatorname{rank}}
\newcommand{\range}{\operatorname{range}}
\newcommand{\trace}{\operatorname{tr}}
\newcommand{\Null}{\operatorname{null}}

\newcommand{\Matlab}{\textsc{Matlab}\xspace}
\newcommand{\Octave}{\textsc{Octave}\xspace}
\newcommand{\pylab}{\textsc{pylab}\xspace}
\newcommand{\longMOP}{\textsc{Matlab}\big|\textsc{Octave}\big|\textsc{pylab}\xspace}
\newcommand{\MOP}{\textsc{M}\big|\textsc{O}\big|\textsc{p}\xspace}

\newcommand{\prob}[1]{\bigskip\noindent\large\textbf{#1.} \normalsize}
\newcommand{\bookprob}[1]{\bigskip\noindent\large\textbf{Exercise #1.} \normalsize}
\newcommand{\probpart}[1]{\smallskip\noindent\textbf{(#1)}\quad }
\newcommand{\aprobpart}[1]{\textbf{(#1)}\quad }


\begin{document}
\scriptsize \noindent Math 310 Numerical Analysis (Bueler) \hfill 13 October 2019
\thispagestyle{empty}

\bigskip
\Large\textbf{\centerline{Review Guide for Midterm Exam}}

\Large\textbf{\centerline{on Thursday, 17 October 2019}}

\normalsize
\bigskip
The in-class Midterm Exam on Thursday is \emph{closed book} and \emph{closed notes} and \emph{no calculator}.  (Bring only a writing implement.)  I will aim for an exam which takes 60 minutes to do, but you will have the whole 90 minute period if you want that.  You can leave the room once you are done (and you have double-checked your answers).

\medskip
I encourage you to work with other students on this Review Guide.  Read the relevant parts of the textbook\footnote{Greenbaum \& Chartier, \emph{Numerical Methods: Design, Analysis, and Computer Implementation of Algorithms}, Princeton University Press 2012.} and identify the first items you \emph{don't} understand as you get to them.  Talk about that!  If you come prepared it will be easy.

\medskip
Problems will be in these categories:
\begin{itemize}
\item apply an algorithm/method in a simple concrete case,

\emph{E.g.~Do two steps of bisection on this problem.}

\item state a theorem or definition,

\emph{E.g.~State Taylor's Theorem.  (I will not ask you to \emph{prove} theorems.  Two theorems you should memorize are listed below.)}

\item write a short pseudocode or \Matlab code to state an algorithm,

\emph{E.g.~Write Newton's method as a \Matlab code or pseudocode.}

\item explain/show in words, and

\emph{E.g.~Why is one of these methods better than another, when applied to this example?  (Write in complete sentences.)}

\item derive an algorithm.

\emph{E.g.~Derive Newton's method.  Also draw a sketch which illustrates one step.}
\end{itemize}

\newcommand{\sgpart}[1]{\bigskip\noindent\large\textbf{#1.} \normalsize}

\sgpart{Sections} See these textbook sections which we covered in lecture and homework:

\qquad 2.1--2.10, \quad 4.1--4.5, \quad 5.2--5.5, \quad 7.1, 7.2.1--7.2.3

\noindent Also read the Chapter introductions for Chapters 2, 4, 5, and 7.  You can omit Chapter 3 and 6 entirely, for now, but rereading Chapter 1 is not a bad idea.  Reading sections 5.1, 5.6, and 5.7 is mostly unnecessary but might be interesting.  For now you can stop reading at the start of section 7.2.4.

\sgpart{Definitions}  Please be able to use these words correctly and/or write a definition when requested.
  \begin{itemize}
  \item  the \emph{absolute error} of $\hat y$, a computed quantity, versus the exact value $y$ is $|\hat y - y|$
  \item  the \emph{relative error} of $\hat y$ versus the nonzero exact value $y$ is $|\hat y - y|/|y|$  
  \item  \emph{fixed point} and \emph{fixed point iteration}  (section 4.5)
  \end{itemize}
\vfill

\newpage
\sgpart{Theorems}  You should understand the statements of these theorems, and be able to apply them in particular cases.  I will not ask you for the proofs.
  \begin{itemize}
  \item  Intermediate Value Theorem (Thm 4.1.1) \qquad  \textsc{Memorize}
  \item  Taylor's theorem with remainder (Thm 4.2.1) \qquad  \textsc{Memorize}
  \item  Newton's method converges quadratically theorem (Thm 4.3.1)
  \item  fixed point convergence theorem (Thm 4.5.1)
  \end{itemize}

\sgpart{Algorithms}  You need to be able to recall these algorithms from memory, or re-derive them as needed.
  \begin{itemize}
  \item  bisection method (section 4.1)
  \item  Newton's method (section 4.3)
  \item  secant method (section 4.4.3)
  \item  Gaussian elimination to solve linear systems (section 7.2)
  \item  forward substitution to solve lower-triangular systems (subsection 7.2.2)
  \item  back substitution to solve upper-triangular systems (section 7.2)
  \item  Gaussian elimination \emph{as LU decomposition} (section 7.2)
  \item  Gaussian elimination \emph{with partial pivoting} (subsection 7.2.3)
  \end{itemize}
Your \underline{three key concerns} about algorithms should be:
\begin{enumerate}
\item What problem does it solve?
\item Can I run the algorithm by hand in small cases with nice/convenient numbers?
\item How does it compare to the other algorithms which solve similar/same problems?
\end{enumerate}

\sgpart{Concepts}
  \begin{itemize}
  \item anonymous functions in \Matlab (section 2.8)
  \item number of steps $k$ for bisection to reduce interval size to $2\delta$ (section 4.1, p.~78)
  \item floating-point representation and IEEE double precision (sections 5.3 \& 5.4)
  \item row operations as left multiplication by lower-triangular matrices (section 7.2)
  \item counting operations (subsection 7.2.1)
  \item using a factorization $A=LU$ to solve $A\bx=\bb$ by two triangular solves (section 7.2)
  \end{itemize}
For a given floating-point system you should \emph{understand} a description of the bit representation and \emph{be able to find} the machine precision $\eps$, the largest representable number, and the smallest positive (normal) representable number.

\end{document}

